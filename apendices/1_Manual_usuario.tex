\section{Manual de Usuario}

\subsection{Material necesario}

\quad A parte de la aplicación descargada, el usuario debe disponer de los siguientes elementos:
\begin{itemize}
	\item Una cardboard o similar
	\item Unos cascos estéreo
	\item Un móvil con sistema mínimo \textit{android 4.4 Kit Kat}
\end{itemize}

\subsection{Acciones del jugador}

\quad Las acciones que el usuario puede desarrollar son las siguientes;

\begin{itemize}
	\item Andar hacia adelante: presionar el botón de la cardboard.
	\item Girar la cámara: girar sobre uno mismo para que el acelerómetro determine hacia dónde miras.
	\item Interaccionar con un elemento: si se puede interaccionar, al apuntar la retícula hacia el esta cambiará a una barra de carga que, al terminar, desata la interacción asociada.
\end{itemize}

\subsection{Objetivos en las pantallas}

\subsubsection{Init}

\quad Esta es la primera escena de la aplicación y está pensada para acostumbrarse a la interfaz de usuario.\\

\quad La tarea es bien sencilla en este caso: encontrar al ser invisible que nos está llamando desde algún lugar.\\ 

\quad Cuando por fin lo encontremos, nos mandará a otra escena después de despedirse.\\

\subsubsection{EchoChamber}

\quad Aparecemos en una habitación cúbica después de despedirnos del amigo invisible donde suena una tonadilla alegre.\\

\quad En esta zona hay tres particularidades:
\begin{itemize}
	\item Un icosaedro
	\item Un menú para cerrar la aplicación o avanzar a la siguiente zona
	\item Un menú que indica cambiar el material  
\end{itemize}

\quad ¿No hay mucho eco aquí? El usuario deberá interaccionar con los distintos elementos y comprobar qué efectos tienen sus acciones en ese entorno.  

\subsubsection{EchoChamber}

\quad Esta ya es la última zona, un pequeño bosque con distintas aves.Prueba a acercarte e interaccionar con algunas, quizás interaccionar con ellas modifique el cartel en el que se lee “Most Wanted!!”.\\

\newpage




