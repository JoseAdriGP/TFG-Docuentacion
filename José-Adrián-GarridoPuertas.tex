\input{preambuloSimple.tex}
\usepackage{url}

\title{	
	\normalfont \normalsize
	\begin{figure}[htb]
		\centering
		\includegraphics[width=0.25\textwidth]{./imagenes/1}
	\end{figure}
	\textsc{\textbf{TRABAJO FIN DE GRADO} \\ Grado en Ingeniería Informática \\ 
	Curso 2018-2019} \\ [25pt] 
	\horrule{0.5pt} \\[0.4cm]
	\huge VR8D \\
	\huge Aplicando sonido 8D a las tecnologías de realidad virtual
	\\ 
	\horrule{2pt} \\[0.5cm]
	\textbf{Autor}\\ {José Adrián Garrido Puertas}\\[1.0ex]
	\textbf{Director}\\ {Marcelino Cabrera Cuevas}\\[0.5cm]
	\includegraphics[width=0.3\textwidth]{imagenes/etsiit_logo.png}\\[0.1cm]
	\date{\normalsize\today} 
}



%----------------------------------------------------------------------------------------
% DOCUMENTO
%----------------------------------------------------------------------------------------

\begin{document}
	
	\maketitle % Muestra el Título
	
	\thispagestyle{empty} 	% Tres paginas en blanco
	\textcolor[rgb]{1.00,1.00,1.00}{palabra} 
	\newpage %inserta un salto de página
	\thispagestyle{empty} 
	\textcolor[rgb]{1.00,1.00,1.00}{.}  
	\newpage %inserta un salto de página
	\thispagestyle{empty} 
	\textcolor[rgb]{1.00,1.00,1.00}{.} 
	\newpage %inserta un salto de página
	
	%--------------------------------------
	% PORTADA
	%--------------------------------------
	
	
\centering

\vspace{3.3cm}
\begin{figure}[htb]
	\centering
	\includegraphics[width=0.25\textwidth]{./imagenes/logo.png}
\end{figure}
\vspace{0.5cm}
{\Huge\bfseries VR8D\\}
\noindent\rule[-1ex]{\textwidth}{3pt}\\[3.5ex]
{\large\bfseries Aplicando sonido 8D a las tecnologías de realidad virtual\\[4cm]}
\textbf{Autor}\\ {José Adrián Garrido Puertas}\\[1.0ex]
\textbf{Director}\\ {Marcelino Cabrera Cuevas}\\[0.5cm]
	
	\thispagestyle{empty} 
	\textcolor[rgb]{1.00,1.00,1.00}{.} 
	\newpage %inserta un salto de página
	
	%--------------------------------------
	% PREFACIO ESP
	%--------------------------------------
	
	\begin{center}
{\large\bfseries VR8D: aplicando sonido 8D a las tecnologías de realidad virtual}\\
\end{center}

\begin{center}
José Adrián Garrido Puertas\\
\end{center}

\begin{flushleft}
	\noindent{\textbf{Palabras clave}: VR, sonido, 8D, Unity, cardboard, eco, posicionamiento, inmersión, oclusion culling, shaders ......}\\
	
	\vspace{0.7cm}
	\noindent{\textbf{Resumen\vspace{0.5cm}}}\\
	En la actualidad, se pueden observar gran variedad de estudios y trabajos relacionados con el concepto de inmersión, buscando con ello facilitar la asimiliación por parte de los usuarios de la información que el programa, terapia, estudio... proporcionan.   
	Con este fin, este proyecto se dispone a investigar la integración en la realidad virtual de los algoritmos de sonido 8D para mejorar la inmersión del usuario en un entorno virtual.
	El trabajo presentado ha sido montado en unity, desarrollándolo como una aplicación en android.
	Las interacciones del usuario con el entorno virtual solo requieren la aplicación, un movil con una cardbord y uno auriculares estereo.
\end{flushleft}

\newpage %inserta un salto de página



	
	\thispagestyle{empty} 
	\textcolor[rgb]{1.00,1.00,1.00}{.} 
	\newpage %inserta un salto de página
	
	%--------------------------------------
	% PREFACIO ING
	%--------------------------------------
	
	\thispagestyle{empty}

\begin{center}
	{\large\bfseries VR8D: applying 8D sound to virtual reality technologies }\\
\end{center}

\begin{center}
	José Adrián Garrido Puertas\\
\end{center}

\begin{flushleft}
	\noindent{\textbf{Keywords}: Vr, sound, Unity, cardboard, eco, positioning, immersion, oclusion culling, shaders ......}\\
	
	\vspace{0.7cm}
	\noindent{\textbf{Abstract\vspace{0.5cm}}}\\
	At present, a wide variety of studies and works related to the concept of immersion can be observed, seeking to facilitate the assimilation by users of the information that the program, therapy, study... provide.
	To this end, this project is preparing to investigate the integration into virtual reality of the 8D sound algorithms to improve user immersion in a virtual environment.
	The work presented has been assembled in unity, developing it as an application in android.
	The user's interactions with the virtual environment only require the application, a mobile with a cardbord and a stereo headset.
\end{flushleft}

\newpage %inserta un salto de página


	
	\thispagestyle{empty} 
	\textcolor[rgb]{1.00,1.00,1.00}{.} 
	\newpage %inserta un salto de página
	
	%--------------------------------------
	% CONSENTIMIENTO
	%--------------------------------------
	
	\input{prefacios/consentimiento}
	
	\thispagestyle{empty} 
	\textcolor[rgb]{1.00,1.00,1.00}{.} 
	\newpage %inserta un salto de página
	
	%--------------------------------------
	% FIRMA TUTOR
	%--------------------------------------
	
	\noindent\rule[-1ex]{\textwidth}{2pt}\\[4.5ex]

D. \textbf{Marcelino Cabrera Cuevas}, Profesor del Área de XXXX del Departamento YYYY de la Universidad de Granada.

\vspace{0.5cm}

\textbf{Informan:}

\vspace{0.5cm}

Que el presente trabajo, titulado \textit{\textbf{VR8D: aplicando sonido 8D a las tecnologías de realidad virtual}},
ha sido realizado bajo su supervisión por \textbf{José Adrián Garrido Puertas}, y autorizamos la defensa de dicho trabajo ante el tribunal que corresponda.

\vspace{0.5cm}

Y para que conste, expiden y firman el presente informe en Granada a 30 de Agosto de 2019 .

\vspace{1cm}

\textbf{Los directores:}

\vspace{5cm}

\noindent \textbf{Marcelino Cabrera Cuevas}

\newpage %inserta un salto de página

	
	\thispagestyle{empty} 
	\textcolor[rgb]{1.00,1.00,1.00}{.} 
	\newpage %inserta un salto de página
	
	%--------------------------------------
	% AGRADECIMIENTOS
	%--------------------------------------
	
	\thispagestyle{empty}

\begin{flushleft}
	\textbf{\LARGE Agradecimientos}\\
\end{flushleft}

\vspace{1cm}

\begin{flushleft}
	Agradecimientos a mi tutor Marcelino Cabrera Cuevas, sin el cuál nunca se me habría ocurrido buscar un tfg que planteara algo distinto a la típica interacción con un ordenador de teclado-ratón.\\
	Agradecimientos a mis padres y hermana, ya que sin su constante apoyo no habría podido llevar a cabo mis estudios ni sacar adelante este proyecto.\\
	Agradecimientos a Juan Hernández García, quien presta su voz para la primera escena de la aplicación y quien es un inestimable amigo.\\
	Agradecimientos a Rubén Jiménez Martínez, Vanesa Rodríguez Rodríguez, Nathaniel Jiménez Rodríguez y Juán Manuel Olivero Maroto, cuya amistad ha sido un importante apoyo durante estos años\\
	Agradecimientos a José Pedro Cirre Mateos, David Galindo López, Alén Blanco Domínguez, Michaelle López Eudaric, Benjamín Alba Morales, Miguel Ángel Cano Mesa, Raúl Alberto Calderón López y David Padilla Montero,  quienes son antiguos alumnos y grandes amigos.\\
	Agradecimientos a José María Esteo Christopoulo, antiguo estudiante de bellas artes y gran amigo con el que he trabajado en unity anteriormente\\
	Agradecimientos a aquellos profesores que se han dedicado a nosotros sus estudiantes, alentandonos a mantener la curiosidad sobre la profesión que hemos escogido.\\
	Agradecimientos a mis compañeros de carrera, tanto los que han terminado como los que no, ya que hemos compartidos aulas, alegrías, estrés y sobre todos muchos cafés.\\
	Agradecimientos a la universidad, que aunque con el paso de los años sientes que te ha quitado mucho, al terminar te das cuentas que te ha dado mucho más.\\
\end{flushleft}

\newpage %inserta un salto de página

	
	\thispagestyle{empty} 
	\textcolor[rgb]{1.00,1.00,1.00}{.} 
	\newpage %inserta un salto de página
	
	%--------------------------------------
	% INDICES
	%--------------------------------------
	
	\tableofcontents % para generar el índice de contenidos
	
	\listoffigures % para generar índice de imágenes.
	
	%\listoftables % para generar índice de tablas.
	
	\lstlistoflistings

	\newpage
	
	%--------------------------------------
	% INTRODUCCION
	%--------------------------------------
	
	\section{Introducción}

\subsection{Contexto}
\justify
\quad El contexto en el que vivimos en la actualidad se presenta como un marco en constante cambio, requiriendo de esta forma que las personas sean capaces de adaptarse mejor y más rápidamente a una gran variedad de situaciones.\\

\quad En el caso de aquellas personas que trabajan en el sector tecnológico esta situación siempre se ha dado, pero lo cierto es que los últimos años se ha incrementado, teniendo por ejemplo que aprender en el menor tiempo posible un lenguaje de programación, preparar una presentación para un proyecto…\\

\quad Esto nos ha llevado a buscar nuevos conceptos y métodos para poder sobreponernos a una situación que es una gran fuente de estrés, y aunque se han planteado gran cantidad de soluciones que pasan desde algo tan simple como aprender a gestionar tu tiempo de forma eficiente, hasta métodos de aprendizaje alternativos, lo cierto es que la tecnología puede ayudar de una forma mucho más activa de lo que podría parecer a simple vista.\\ 

\quad También hay que remarcar que cada vez se busca poder trabajar de forma remota, ya sea como en el caso de los desarrolladores que pueden estar en constante contacto con miembros del equipo en distintas parte del mundo, conferencias por parte de distintas personas, trabajos conjuntos de arte mediante internet o incluso  llegando a realizarse operaciones a distancia \cite{BBC}. Todo esto requiere que el servicio sea lo más rápido posible, pero cada vez más se exige también que las posibilidades que abarque aumenten de manera gradual.\\ 

\quad Otro tema que también es interesante abordar es el videojuego, que ha demostrado ser una herramienta muy valiosa a la hora del aprendizaje, quedando patente en casos de gente que conoce a la perfección las diferentes etapas del videojuego al que dediquen en ese momento, llegando a saber ingentes cantidades de información de un mundo virtual, basado o no en el nuestro.\\

\quad Hay muchos factores que facilitan la asimilación de información, como la pasión por el tema tratado o lo amena que la información se distribuya de modo que el usuario pueda consumirla en pequeñas dosis, pero un concepto que no empezó a tratarse hasta hace relativamente poco fue el de la \textit{inmersión}, que a priori parece un concepto realmente potente que podría darnos una solución duradera. \\

\quad Todo lo anteriormente mencionado pide una solución totalmente innovadora… o quizás no tan innovadora. Pensándolo fríamente hay buenas ideas en el pasado que quizás no se exploraron por la limitación de la tecnología de la época. Esto ya se está viendo con ideas como el \textit{raytracing}, cuyo fue planteado en 1980 por Turner Whitted.\\
\quad A mediados del siglo XX se empezaron a ver aparatos para visualizar fotografías en 3D llamados \textit{View-Master} o una patente de 1957 para unas gafas de realidad virtual \cite{Tech}, por no mencionar que uso de entornos completamente virtuales para entrenamientos militares en la aviación lleva con nosotros desde la década de los 70\cite{NCSA}, suceso que demostró que no es necesario realizar una práctica con un avión real para obtener los conocimientos requeridos.\\

\quad Basándome en lo anteriormente expuesto, puedo decir que quizás el concepto de entorno virtual pueda ser de gran ayuda, pero está claro que por sí solo no termina de completar una idea que pueda ser funcional y a la vez no haya ya sido explotada.\\

\subsection{Motivación}
\quad La motivación que me lleva a este trabajo no es otra que contribuir a lo anteriormente expuesto. La tecnología avanza inexorablemente y no para de incorporar conceptos de diferentes doctrinas, y es necesario que los desarrolladores dediquemos tiempo para poner ideas nuevas sobre la mesa, o al menos escoger ideas anteriores y explotar su verdadero potencial.\\

\quad Ante esto, me plantee rescatar el concepto de realidad virtual y unirlo los conceptos de gamificación\cite{Educativa}\cite{GameChina} e inmersión en el entorno.\\

\quad Ahora solo nos queda encontrar una forma de aplicarlos dentro de un entorno virtual, de forma que lo primero que necesitaríamos sería un objetivo para cumplir con la gamificación y alguna forma de aumentar la inmersión entre los distintos escenarios que pudiéramos encontrar en ella.\\

\quad Lo primero que puede venir a la cabeza es mejorar los gráficos en la aplicación, pero lo cierto es que vamos a trabajar con un dispositivo móvil ya que dispositivos como Oculus Rift S cuestan alrededor de 250 euros\cite{OcuRS}.\\

\quad Ante este mercado hay que plantear otra solución que no aumente excesivamente el procesado en la aplicación o requiera de equipos de elevado precio,\\

\quad La solución por la que opté al final fue el sonido 8D. La idea de utilizar esta tecnología surgió a raíz de un video sobre ella por Jaime Altozano \cite{JAlto}, y aunque parezca una tecnología revolucionaria, en realidad tiene ya un tiempo y una serie de problemas \cite{8D} que quizás sí podamos ser capaces de solucionar con la tecnología actual.

\subsection{Objetivo}
\quad Ya tenemos las herramientas conceptuales, pero hay que detenerse un poco y analizar cuál o cuáles son los objetivos que se pretenden uniendo esta serie de tecnologías.\\

\quad El primer objetivo por supuesto hacer la modificaciones pertinentes en el algoritmo de sonido 8D con la intención de mejorarlo e introducirlo dentro de una aplicación de realidad virtual.\\

\quad Otro objetivo es observar el concepto de inmersión por parte del usuario al tener distintos focos de sonido que se sitúen a su alrededor, ya sean estáticos o tengan algún tipo de de movimiento asignado.\\

\quad Como último objetivo se encuentra la interacción auditiva por parte del usuario y como esta cambia la experiencia del usuario.\\

\newpage







	\thispagestyle{empty} 
	\textcolor[rgb]{1.00,1.00,1.00}{.} 
	\newpage %inserta un salto de página
	
	%--------------------------------------
	% ESPECIFICACION DE REQUISITOS
	%--------------------------------------
	
	\section{Especificaciones y Requisitos}

\subsection{Descripción del proyecto}
\subsubsection{Introducción}
\quad Esta sección va a hablar los requisitos principales para el desarrollo de una aplicación que haga uso de los algoritmos de 8D presentados en la actualidad y aplicarlos en un entorno de realidad virtual.\\

\quad El caso presentado comparte similitudes con el desarrollo de un videojuego, las cuales son debidas a que la aplicación en cuestión es un entorno interactivo para poder testear el algoritmo de sonido 8D. Esto se desarrolla en los siguientes apartados.

\subsubsection{Equipo de desarrollo}
\quad Para proyectos de mayor envergadura, el desarrollo de una aplicación de estas características requiere un equipo conformado por los siguientes integrantes:\\
\begin{outline}
\1 Gestión/Producción:
	\2 CEO
	\2 Directores de proyecto
	\2 Productores	
\1 Diseño:
	\2 Game designer
	\2 Level designer
\1 Arte y animación:
	\2 2D artist:
		\3 Concept artist
		\3 Pixel artist
		\3 UI artist
	\2 3D artist (modelaje, iluminación, texturización...):
		\3 Personajes
		\3 Escenario
\1 Animación:
	\2 Rigging
	\2 Animator
\1 Sonido:
	\2 Ingeniero de sonido
	\2 Compositor
\1 Programación:
	\2 Backend
	\2 Frontend
\1 Quality Assurance (QA):
	\2 Testers
	\2 Control de calidad
\end{outline}

\quad En este caso, el equipo está concentrado en una sola persona trabajando todos los aspectos, de forma que se recurre también a assets gratuitos para poder suplir las disciplinas de las que no se tiene conocimiento.\\

\subsubsection{Comparativa de motores gráficos}

\quad Uno de los aspectos más importantes en el desarrollo de esta aplicación es contar con un framework 4 o un motor gráfico sobre el cual poder desarrollarla. A lo largo de
los últimos años se han estandarizado una serie de productos que facilitan este desarrollo.\\

\quad Por ello, se estudiarán diferentes opciones disponibles, teniendo en cuenta que no se pretende señalar todos y cada uno de los motores gráficos del mercado sino una
selección de ellos de acuerdo a mis intenciones de abarcar el máximo espectro posible sin citar a todos.\\

\quad A continuación, se señalan algunos casos:

\begin{itemize}
\item{\textbf{Source 2 Engine}}

\quad Sucesor del motor gráfico Source propiedad de Valve, y motor de varios juegos famosos como Portal o Half Life. Estará disponible de forma pública y gratuita siempre y cuando se publique para la plataforma de Valve, Steam. Compatible con Vulkan (OpenGL) y usa un motor de físicas propio llamado Rubikon que sustituye a Havok. \\

\quad Algunos de los juegos realizados con él son Counter-Strike: Global Offensive y Dota 2.\\

\item{\textbf{Unity}}

\quad Disponible para Windows, OS X y Linux, históricamente asociado a juegos de menor presupuesto, juegos indie y de móviles.\\

\quad Dispone de varias versiones:
\begin{itemize}
	\item Unity Personal: gratuita si no se sobrepasan los 100 mil dólares de ingresos
	\item Unity Plus: suscripción de 35 dólares al año, enfocado a desarrolladores móviles con ingresos menores a 200 mil dólares y que incluyen algunos servicios de Unity
	\item Unity Professional: suscripción de 125 dólares al mes, sin límite de ingresos, con todos los servicios de Unity
	\item Unity Enterprise
	\item Unity Educational
\end{itemize}
\quad Las versiones Pro y Plus ofrecen soporte a la versión y acceso a todas las actualizaciones. También cabe señalar la Asset Store, lugar donde se concentran extensiones, herramientas y assets para los usuarios tanto gratuitos como de pago.\\

\quad Algunos juegos hechos con Unity son Wasteland 2, Pillars of Eternity, Hearthstone o Firewatch.\\

\item{\textbf{CryEngine}}

\quad Desarrollado por Crytek y usado por primera vez en el juego Far Cry. La versión 5 utiliza de forma nativa DirectX12, Vulkan y soporte para VR.\\

\quad Introdujo un nuevo modelo de licencias, el “paga lo que quieras”, 100\% royalty-free en la actualidad, para las plataformas Windows, Linux, PlayStation 4, Xbox One, Oculus Rift, HTC-Vive, Open-Source VR y PlayStation VR.\\

\quad También posee un bazar, el “CRYENGINE Marketplace” donde los beneficios de las ventas son de un 70\% para el motor y el restante 30\% para el desarrollador del contenido.\\

\quad Juegos que usan este motor: Saga Crysis, Ryse: Son of Rome, Aion Online (MMORPG online).\\

\item{\textbf{Unreal Engine}}

\quad Sucesor de Unreal Development Kit (UDK), propiedad de Epic Games. Gratuito en la actualidad, aunque se paga a la empresa un 5\% de los beneficios cada trimestre a partir del momento en que el juego gane sus primeros 3000 dólares. Desarrollado en C++, para su uso en plataformas como Windows, OS X, Linux, iOS, Android, PlayStation 4, Xbox One, Nintendo Switch y navegadores (HTML5). También tiene soporte a Vulkan.\\

\quad Al igual que Unity, tiene su propio bazar llamado “Unreal Engine Marketplace”, donde permite comprar y vender contenido (desde modelos, a tutoriales pasando por sonidos, efectos especiales, etcétera). También ha puesto en marcha un programa llamado “Unreal Dev Grants” con un presupuesto de cinco millones de dólares destinado a financiar a los desarrolladores que presenten proyectos innovadores usando el motor. Probablemente sea el motor más usado en la actualidad.\\

\quad Juegos que usan este motor: DMC (Devil May Cry de Ninja Theory), saga Batman Arkham.\\

\item{\textbf{Snowdrop}}

\quad Motor propiedad de Ubisoft creado por Massive Entertainment (Ubisoft). Codificado en C++, tardó cinco años en ser desarrollado.\\

\quad Su punto fuerte es la iluminación y el sistema de destrucción. Utiliza el motor de físicas Havok. Este motor está en esta lista porque a pesar de ser uno de los mejores del mercado no está disponible para todo el mundo: Sólo los equipos de desarrollo de Ubisoft tienen acceso a este motor.\\

\quad Juegos: Tom Clancy's: The division, Mario + Rabbids Kingdom Battle, Skull \& Bones.\\

\item{\textbf{Frostbite}}

\quad Desarrollado por DICE (Electronic Arts) y diseñado para ser un motor exclusivo de EA. \\

\quad Inicialmente diseñado para hacer juegos en primera persona, ha ido evolucionando abrazando otros géneros. Codificado en C++, C\#, Lua y IronPython. Las plataformas objetivo son PC, PlayStation 4 y Xbox One. Enfocado a permitir una mayor escala de interacciones multijugador en escenarios dinámicamente destruibles con condiciones atmosféricas cambiantes. \\

\quad Como Snowdrop, es un motor propietario. Su presencia aquí es para poder comparar dos motores propietarios. \\

\quad Juegos: Battlefield, Need for speed, Dragon Age, Mirror's Edge, FIFA, Star Wars: Battlefront\\

\item{\textbf{Amazon Lumberyard}}

\quad Motor propiedad de Amazon, gratuito y orientado a juegos AAA (grandes producciones). \\

\quad Basado en CryEngine, está programado en C++ y en Lua. Tiene como características principales la integración con Amazon Web Services (AWS) y Twitch. El código fuente es completo y gratuito y no hay cuotas de suscripción ni requisitos económicos. \\

\quad Solamente hay que pagar por los servicios de AWS que se utilicen (así es como sacan beneficios). Las plataformas objetivo son Windows, PlayStation 4, Xbox One, iOs, Android (con soporte limitado en estas dos últimas), Oculus Rift, HTC-Vive, OpenSource VR y PlayStation VR.\\

\quad Juegos: Star Citizen\\

\item{\textbf{UbiArt framework}}

\quad Motor gráfico propiedad de Ubisoft y diseñado por el creador de Rayman, Michel Ancel. Totalmente centrado en la creación de proyectos 2D y 2.5D, responde a una
búsqueda de la compañía de facilitar el desarrollo de juegos a un equipo pequeño de personas y con un presupuesto reducido.\\

\quad Su punto fuerte es la facilidad para crear animaciones que da a los artistas y el aspecto artístico presente en los títulos desarrollados con este motor.\\

\quad Juegos: Rayman Origins, Rayman Legends, Valiant Hearts, Child of light \\

\item{\textbf{GameMaker}}

\quad Propiedad de YoYo Games y diseñado para permitir a usuarios sin conocimientos de programación desarrollar juegos fácilmente. Contiene un lenguaje de programación de scripts, Game Maker Language (GML), para usuarios experimentados. Licencia EULA.\\

\quad Juegos: Saga Hotline Miami, Undertale.\\
\end{itemize}

\subsubsection{Decisión sobre el motor}
\quad Finalmente se ha optado por Unity, principalmente por mi anterior experiencia con él, trabajando en \textit{Game Jams} como en proyectos personales. Además presenta una gran variedad de assets gratuitos con los que complementar mi proyecto, así como un amplia documentación tanto en su página principal como en foros.

\subsection{Requisitos}

\quad El servicio que esta aplicación proporciona es simple, ya que se compone de tres entornos virtuales que por el que poder moverte e interaccionar, pero debemos tener en cuenta una serie de requisitos indispensables para el correcto funcionamiento de esta.\\

\quad Dicho esto, nos disponemos a enumerar y describir dichos requisitos para una mejor comprensión por parte del lector de las intenciones de esta aplicación.\\

\subsubsection{Funcionales}

\quad Aquí nos disponemos a hablar de todos aquellos requisitos que la aplicación debe proporcionar directamente. En nuestro caso, estos requisitos son:

\begin{itemize}
	\item Entornos virtuales diferenciados entre sí
	\item Sonido preparado mediante los algoritmos 8D
	\item Interfaz adecuada a la situación planteada
\end{itemize}

\quad Aunque sean pocos, es importante tener en cuenta que cada uno de ellos es imprescindible a la hora del correcto funcionamiento de la aplicación, y por consiguiente, no cumplir uno de ellos invalida la aplicación de forma automática. Así pues, desarrollarlos un adecuadamente.\\

\quad Cuando se habla del \textit{entorno virtual} se quiere hacer referencia al hecho de que necesitamos un entorno por el que el usuario pueda moverse y trabajar con los elementos dentro del entorno. Es importante remarcar que en este proyecto, los entornos van aumentando su complejidad gráfica e interna para así desarrollar una sensación de progresión en el usuario.\\

\quad El sonido es imprescindible para el usuario, ya que proporciona el feedback que se requiere de la aplicación para poder interactuar con los elementos de cada escena. Si el sonido funciona mal, o directamente no funciona, esta aplicación puede considerarse un auténtico fracaso.\\

\quad Al hablar de una interfaz adecuada, se hace referencia al hecho debe cumplir con su propósito y proporcionar feedback al usuario sobre qué se está haciendo con ella. Con respecto a la interfaz hay otros conceptos a tener en cuenta, pero ya entran dentro de los requisitos no funcionales, por lo que los desarrollaremos siguiente apartado,\\

\subsubsection{No funcionales}

\quad Analizar los requisitos que la aplicación debe proporcionar de forma no directa puede ser un poco más problemático, porque dentro de ella se quieren integrar conceptos más abstractos.\\
\quad Para hablar de estos requisitos, vamos a hacer una división sobre el objetivo que pretenden.

\subsubsubsection{Requisitos relacionados con el concepto de inmersión}

\quad Esta sección plantea aquellos requisitos que busca hacer que la experiencia con la aplicación sea lo más inmersiva posible, de forma que el usuario pueda “olvidar” en su subconsciente que se encuentra en un entorno virtual.\\

\quad Conseguir esto requiere varios factores a tener en cuenta, pero el primero sobre el que debemos hablar es la tasa de frames, la cual deberá ser estable dentro de las diferentes escenas. La estabilidad predominará sobre la cantidad de frames ya que se va a trabajar con un smartphone antiguo, y una gran cantidad de frames implicaría también reducir el aspecto gráfico de la aplicación.\\

\quad Otro aspecto a tener en cuenta son los tiempos de carga. El planteamiento de esta aplicación es tener unos tiempos de carga prácticamente inexistentes, de forma que el usuario no sienta una desconexión abisal entre las diferentes escenas.\\

\quad Aquí se vuelve a hablar de los menús, ya que estos no deben estar integrados en el visor del usuario, si no que deben ser elementos que encontremos por el entorno. Este planteamiento se debe a que la intención no es simular que el usuario lleva un casco con una interfaz, si no que se pretende simular que el usuario realmente se encuentra en las escenas presentadas. Un menú que interfiera con la visión rompería la ilusión de verse en otro lugar distinto a la habitación donde se está probando la aplicación.\\

\quad El sonido 8D tiene también un valor añadido en los requisitos no funcionales, ya que este debe proporcionar la posición del emisor de una forma clara, de forma que el usuario sepa en todo momento dónde se sitúa el foco emisor. Esto será útil pues varios de los emisores que se encontrarán en la escena serán también elementos con los que poder interaccionar.\\

\subsubsubsection{Requisitos relacionados con el concepto de gamificación}

\quad Se pretende que parte de las interacciones se produzcan surgidas de un interés relacionado con la gamificación. De está forma, el usuario aprenderá a moverse y a interaccionar con los distintos elementos que componen la escena tomando como referencia los videojuegos.\\



\newpage



	\thispagestyle{empty} 
	\textcolor[rgb]{1.00,1.00,1.00}{.} 
	\newpage %inserta un salto de página
	
	%--------------------------------------
	% PLANIFICACION
	%--------------------------------------
	
	\section{Planificación General}

\subsection{Introducción}
\quad En esta sección se describe la planificación temporal del proyecto, así como el beneficio monetario que se espera de la aplicación.\\

\subsection{Planificación}

\quad El desarrollo de este proyecto se inició seriamente en Marzo de 2019, pero debido a asuntos personales del autor, no se pudo dedicar todo el tiempo necesario desde el mismo inicio, lo que provocó que la implementación y pruebas se retrasaran en el marco temporal.\\

\quad La planificación temporal de este proyecto se divide en una serie de etapas bien diferenciadas, mostradas en el siguiente diagrama de Gantt:\\

\begin{figure}[htb]
	\centering
	\includegraphics[width=1.2\textwidth]{./imagenes/diagramaGantt}
	\caption{Distribución del tiempo de desarrollo de la aplicación}
\end{figure}

\subsection{Tareas}

\quad Ahora pasamos a repasar las distintas tareas que podemos encontrar a lo largo de todo el periodo de desarrollo, Para facilitar la comprensión las agrupamos según las diferentes etapas del desarrollo.\\

\subsubsection{Etapa de análisis}

\quad Durante esta etapa se tuvo que analizar tanto las herramientas como los  diferentes planteamientos para afrontar el problema al que nos enfrentamos. En concreto, gran parte del tiempo se dedicó a comprender tanto el algoritmo 8D como las diferentes implementaciones de la realidad virtual.\\

\quad Hay que tener en cuenta también que es importante  conocer cómo funciona tanto el 8D como la realidad virtual, pero los conceptos no se quedan solo ahí, ya que una parte fundamental es la inmersión dentro de la aplicación y cómo emplearla de forma natural.\\

\quad Esto provocó un estudio exhaustivo de las diferentes formas de implementar un interfaz de usuario en realidad virtual, así como las mejores maneras de interactuar con ellas.\\

\quad Quizás el análisis que más tiempo llevó fue el del algoritmo 8D, pues requería entender desde cero un concepto nuevo para mi, ya que al principio lo entendía de una manera totalmente errónea. A poriori, este debe de trabajar desde la ubicación del usuario al foco, pero los cálculos y las implementaciones vistas me han demostrado que se hace un enfoque bidireccional, de forma que se pueda calcular el volumen del sonido de una forma más precisa.\\

\quad También cabe destacar que desde el principio planteé la aplicación a dispositivos móviles, ya que cualquier otra opción implicaría inevitablemente una inversión de dinero que no podía asumir.\\

\subsubsection{Etapa de diseño} 

\quad Abordados ya los conceptos que debía tener claros, empecé con el planteamiento de la aplicación.\\

\quad La primera idea fue hacer un entorno en un bosque donde escuchar el sonido de varios pájaros revoloteando, y de hecho este planteamiento es que lleva a la escena tres de la aplicación, pero me di cuenta que entonces dejaría muchos conceptos sin explorar, como el eco en una habitación o interactuar con un sujeto que se dirija a ti específicamente.

\quad Este planteamiento fue el que me llevó a buscar nuevas ideas para las escenas, dando lugar a las escenas uno y dos, donde tendremos que buscar a un objetivo invisible y experimentar con un emisor de sonido en un entorno con eco respectivamente.\\

\quad Parte de esta etapa la dediqué también a buscar diferentes assest que pudieran servir para completar los entornos en lo que se moverá el usuario, haciendo de esta forma que sean más creíbles e inmersivos.\\

\quad Por último me plantee cómo explicar las instrucciones de uso, de forma que para evitar ventanas emergentes que romperían la inmersión, o complejos monólogos que abruman al usuario, decidí explicarme con letras grandes en una de las paredes, ya que se añadiría otro elemento a las escenas, haciendo juego con los menús que pudiéramos encontrar en ellas.\\

\subsubsection{Etapas de Implementación, Testeo \& Evolución} 

\quad Estas tres etapas se presentan de forma casi simultánea durante este desarrollo, por lo que considero imprescindible abordarlas también de esta forma.\\

\quad Empecé implementando el bosque, donde lo principal era:

\begin{itemize}
	\item Crear un terreno irregular para dar una sensación más realista
	\item Utilizar un algoritmo procedural ya implementado para situar árboles e hierba
	\item Situar los elementos que podamos encontrar extra, como las rocas o las estatuas que se encuentran en el bosque
\end{itemize}

\quad Lo siguiente fue trabajar con el asset “living birds” para disponer los diferentes pájaros por la zona de forma aleatoria y automática, así como implementar en ellos el 8D.\\

\quad Cuando esto se llevó a cabo, pase a desarrollarla habitación con eco, que implicó también varias tareas:
 
\begin{itemize}
	\item Crear el entorno
	\item Disponer la zona de eco
	\item Desarrollar el control del usuario
	\item Desarrollar los menús para las diferentes interacciones que se quisieran aplicar a los objetos de la zona
\end{itemize}

\quad Para el entorno utilice un prefab de Google VR, y delimite la zona de eco, donde inserte un icosaedro que emite la canción libre de derechos \textit{“If i had a chicken”}. Esto permitió testear el eco de la zona de una manera más sencilla.\\

\quad Ahora tocaba dedicar un tiempo al usuario, por lo que empecé trabajando con la retícula que sería la forma de interaccionar con el entorno, aunque se presentó un problema a la hora de implementar una barra de carga para que el usuario supiera que que se estaba interactuando con algo. Este problema se expone en el apartado \eqref{7.2}.\\

\quad Ahora tocaba desarrollar un método para poder mover al usuario por la escena, por lo que dispuse un control para hacerlo por mando. Esta implementación acabó siendo desechada ya que requería un periférico extra, dando lugar al control implementado de mantener pulsado para avanzar en la dirección hacia la que mire el usuario.\\

\quad Lo siguiente fue implementar un método para que al interactuar con el icosaedro, éste se teletransporte a otro lugar dentro de los límites de las escena.\\

\quad La última implementación en esta etapa fue un menú para poder avanzar al bosque o cerrar la aplicación, menú que también se utilizará en el bosque más adelante para volver a esta habitación o cerrar la aplicación.\\ 

\quad La primera escena fue paradójicamente la última con la que empecé, aunque el diseño era mucho más sencillo. Me plantee el hecho de que en esta aplicación el gran protagonista es el sonido, por lo que una vez implemente un espacio sencillo compuesto por un plano que hace de suelo y muros invisibles que limitan la posibilidad de moverse, cree un cubo que llama al usuario. Fue entonces cuando me di cuenta de que sería más interesante si el cubo fuera invisible y solo apareciera cuando por fín sea encontrado y se interactúe con el, indicándose mediante otro audio que te manda a otra escena, en este caso la habitación del eco.\\

\quad Lo siguiente fue optimizar el bosque, ya que la carga poligonal era excesiva y el móvil no era capaz de mantenerla abierta. Con este fin, dediqué tiempo a reducir y modificar los shaders, así como también dedicar tiempo a implementar “Oclusion Culling”. De esta forma, el bosque pasó de ser un posible descarte a ser probablemente la zona más interesante de la aplicación.\\

\quad Con todo esto ya implementado, añadí al bosque unos muros invisibles para delimitar la zona de movimiento y se lo enseñe mi tutor para el TFG, con quién estuve hablando sobre las posibilidades, y me recomendó que intentara implementar algún objetivo en el bosque, así como alguna manera de interaccionar con el eco de la habitación.\\

\quad Esto me llevó a implementar interacciones con solo algunos pájaros, así como un indicador encima del menú que nos presentase cuales efectivamente habíamos conseguido localizar.\\

\quad En el caso del eco, simplemente implementé una serie de menús formados por los dropdown de Unity, de forma que pudiéramos cambiar el material que forma cada pared de la habitación, así como suelo y techo. Esta implementación también presentó un problema que detallo en el apartado \eqref{7.4}.\\ 

\newpage


	\thispagestyle{empty} 
	\textcolor[rgb]{1.00,1.00,1.00}{.} 
	\newpage %inserta un salto de página
	
	%--------------------------------------
	% ANALISIS
	%--------------------------------------
	
	\section{Análisis}

\subsection{Introducción}

\quad Aqui se analizarán las diferentes partes del proyecto agenas a la programación, como la investigación asociada al proyecto, o las herrramientas que se va a utilizar en el proyecto.\\ 

\subsection{Estado del arte}

\quad Aqui se van a analizar varios aspectos sobre el proyecto.\\

	\subsubsection{Sonido 8D}
\quad El audio 8D es la sensación de escuchar los sonidos a través de unos cascos en ángulos de trescientos sesenta grados, parecido a la realidad virtual.\\ 

\quad Este término que a resugido gracias a las redes sociales, dista de ser nuevo, ya que antiguamente era conocido como ambisonic, binaural o simplemente sonido 3D.\\

\quad Como estrategia de márketing, algunos músicos sugieren que la música danza alrededor de quien la escucha, provocando que muchos de los temas actuales, y no tan actuales, adquieran este sistema.\\

\quad A pesar de los avances, la sensación real y envolvente esta muy lejos de la realidad, ya que muchos elementos no se tiene en cuenta a la hora de determinar el posicionamiento del objeto. Por ejemplo, la transición entre delante y detrás del usuario pueden por momentos ser indistinguibles, ya que no se tienen de forma alguna en cuenta la orientación de las orejas, o el hecho de que si el usuario no se mueve, el cerebro humano no es capáz de distinguir entre una posición detante o detrás,asignando por defecto una posición delantera hasta que la fuente del sonido o el receptor se muevan, permitiendose entonces que el cerebro si situe en el espacio virtual.\\

\quad Ante este problema un desarrollador puede verse abrumado al principio, pero la solución a ambos problemas es más simple de lo que podría parecer, ya que en el caso de la orientación de las orejas es aplicar un coeficiente de division en los sonidos situados detrás del usuario, que se basa en el índice de ambsorción de sonido de la piel humana. En este caso se ha utilizado el de la goma para simular la piel. El según do problema no tiene una solución software a simple vista, pero teniendo en cuenta que el usuario interaccionará con el entorno moviendo la cabeza y avanzando por la escena, de forma que cambia su posición y horientación con respecto al foco, este problema se maquillara por la ppropia inmersión de la aplicación.\\ 

	\subsubsection{Realidad Virtual}
\quad Cuando hablamos de realidad virtual, hablamos de un entorno de apariencia real generado mediante tecnología informática que da la sensación de inmersión al usuario.\\ 

\quad Algunos centros educativos lo están usando para comprobar su viabilidad, sobre todo en casos de personas con dificultades en el aprendizaje, aunque presenta una serie de inconvenientes, como el coste o el espacio físico necesario para el usuario. A pesar de ello, se tienen grandes expectativas sobre su utilidad.\\

\quad Es importante añadir que en los últimos años hemos sufrido un boom con la aparición de la VR en el mundo del videojuego de una forma serie, sin embargo el precio de los equipos para los usuarios hace que los desarrolladores no inoven en este campo, por lo que si no se vuelven más económicos, algunos "expertos" afirman que podría desaparecer. \\

\quad Esta afirmación parece muy alarmista, ya que la VR no se utiliza solo en el hámbito del videojuego y se han hecho grandes avances para otros campos.\\

	\subsubsection{Gamificación}
\quad Se como el uso de diseños, elementos y características de juegos en contextos totalmente ajenos a estos. Socialmente hablando, especialmente a aquellos asiduos a las redes sociales, con los que comparten elementos como la lealtad del usuario, los logros o el reclutamiento a eventos con métodos atractivos y divertidos, dándoles una sensación de control y motivando su uso.\\

\quad Respecto a la educación, muchos han adoptado este estilo de enseñanza aunque no todos están de acuerdo. Su uso sigue expandiéndose cada vez más.\\

\quad En 2017, el Gobierno chino puso en marcha un enorme proyecto piloto en el ámbito social, basado en una especie de gamificación. Con ello, se cambió la magnitud y escala de lo que entendíamos por ‘gamificación’ , de tal manera que conectó la actividad online de sus ciudadanos con su estrategia social a gran escala, incorporándolos a un sistema de medición de conducta individual con ‘premios’ y ‘castigos’, a partir de un sistema de puntación llamada originalmente ‘crédito social’ que en origen, tenía propósitos comerciales relacionados con recompensas o incentivos.\\

\quad Esto no sería posible sin dos factores, la colaboración de grandes empresas chinas basadas en internet (como Alibaba, Baidu o Tencent) y las nuevas leyes de ciberseguridad chinas, que dan cobertura legal al acceso completo a casi todos los datos personales.\\

\subsection{Tecnologías y herramientas}

	\subsubsection{Lenguaje de programación C\#}

\quad Es un lenguaje de programación orientado a objetos desarrollado y estandarizado por Microsoft como parte de su plataforma .NET.\\

\quad Al ser un lenguaje orientado a objetos, admite los conceptos de encapsulación, herencia y polimorfismo. Todas las variables y métodos, incluido el método Main, el punto de entrada de la aplicación, se encapsulan dentro de las definiciones de clase. Una clase puede heredar directamente de una clase primaria, pero puede implementar cualquier número de interfaces. Los métodos que invalidan los métodos virtuales en una clase primaria requieren la palabra clave override como una manera de evitar redefiniciones accidentales. En C\#, un struct es como una clase sencilla; es un tipo asignado en la pila que puede implementar interfaces pero que no admite herencia.\\

\quad Escoger este lenguaje facilita el trabajo con unity, motivo por el cuál es el seleccionado.\\
	
	\subsubsection{GitKraken y GitHub}
	
\quad El uso de un repositorio y control de versiones es algo fundamental para llevar un control del trabajo en la aplicación, asi como poder volver a versiones anteriores del desarrollo en caso de ser necesario.\\

\quad GitKraken es una herramienta que tiene absolutamente todas las funcionalidades que se pueden llegar a querer en una herramienta para control de versiones con un rendimiento que deja en muy mal lugar al resto de herramientas (como SourceTree o Tortoise Git).\\

\quad Es la única herramienta que tiene versión de pago con su licencia PRO (49\$/año), sólo necesitarás esta licencia para un uso profesional o no comercial, pudiendo realizar tus proyectos personales con la versión gratuita sin ningún tipo de problemas, ya que las funcionalidades son las mismas.\\

\quad En cualquier caso, con la licencia de estudiante gratuita de GitHub se puede tener acceso a todas las funcionalidades, tanto de GitHub como de GitKraken.\\


	\subsubsection{Unity}

\quad Ha sido la opción para el desarrollo de la app. Ya hablamos de él en el apartado \textit{2.1.3. Comparativa de motores gráficos}.
	
	\subsubsection{Cardboard}

\quad Son un visor donde se coloca un teléfono móvil para experimentar una experiencia en realidad virtual.\\

\quad Actualmente google lo vende como experias en VR a bajo costo.\\

\newpage




	\thispagestyle{empty} 
	\textcolor[rgb]{1.00,1.00,1.00}{.} 
	\newpage %inserta un salto de página
	
	%--------------------------------------
	% DISENIO
	%--------------------------------------
	
	\section{Diseño}

\subsection{Visión global de la aplicación}

\quad La aplicación consta de tres escenas:\\

\begin{itemize}
	\item Init: donde la aplicación empieza.
	\item EchoChamber: donde se testeó el efecto del eco.
	\item Forest: donde se prueba el efecto del 8D con varios focos de sonido.
\end{itemize}

\quad Antes de empezar a hablar de cada uno de los elementos particulares de cada escena, es imperativo hablar de los elementos que son comunes en todas ellas. De este modo, en toda escena vamos a encontrar una serie de elementos que son imprescindibles: 
\begin{itemize}
	\item Foco de luz: configurado como luz direccional
	\item RealPlayer: contiene la cámara y la retícula y hace la función de representarnos en el mundo virtual
	\item GvrEditorEmulator: encargado de simular con el ratón el movimiento en modo VR
	\item GvrEventSystem: que se encarga de activar el input de colisión entre el rayo de detección que se emite desde la cámara por la retícula y los objetos con los que podemos interaccionar 
	\item SceneManager: se encarga de gestionar los eventos de salir de la aplicación y cambiar de escena
	item GvrControllerMain: se utiliza para determinar controles más allá del giro de cámara y el click, pero al final este elemento no se llegó a utilizar
\end{itemize}

\quad Ahora se pasa a hablar de las particularidades de cada escena.\\

\subsection{Escenas}
	\subsubsection{Init}
\quad Esta escena pretende poner en situación al usuario, indicando con un texto sobre una pared invisible como moverse y que es lo que tiene que hacer en ella.\\

\quad El objetivo será tan simple como encontrar el foco de sonido que lo está llamando. La voz para el foco ha sido aportada por Juan Hernández García, persona que menciono en los agradecimientos.\\

\begin{figure}[htb]
	\centering
	\includegraphics[width=0.6\textwidth]{./imagenes/initInstructions}
	\caption{Instrucciones de uso}
\end{figure}

\begin{figure}[htb]
	\centering
	\includegraphics[width=0.248\textwidth]{./imagenes/inactiveCube}
	\includegraphics[width=0.3\textwidth]{./imagenes/activeCube}
	\caption{Cubo}
\end{figure}

\subsubsubsection{Componentes de Init}

\quad Dicho todo esto, toca hablar de cada uno de los elementos no comunes de la escena. De este modo, quedan por comentar los siguientes elementos:

\begin{itemize}
	\item Cube: sujeto que que se vuelve visible al encontrar su posición siguiendo el sonido de su llamada
	\item Plane: suelo de la escena
	\item Canvas: contiene un texto en que van unas pequeñas instrucciones
	\item Walls: Se forma con cuatro prismas invisibles que actuarán como muros invisibles para delimitar la zona de movimiento
\end{itemize}

	\subsubsection{EchoChamber}
\quad Esta habitación tiene como objetivo mostrar el efecto del eco ante un foco de sonido, en este caso será un icosaedro que toca la tonadilla libre de derechos \textit{If i had a chicken}, típica canción de taberna del oeste. Este icosaedro se tele transportará a otro lugar de la habitación cuando interactuemos con él.\\

\begin{figure}[htb]
	\centering
	\includegraphics[width=0.4\textwidth]{./imagenes/icosaedroInactivo}
	\includegraphics[width=0.4229\textwidth]{./imagenes/icosaedroActivo}
	\caption{Icosaedro sonoro}
\end{figure}

\quad Además, se encontrarán dos menús, unos con dos botones y otro con varios desplegables.\\

\quad Mediante el menú de dos botones se podrá salir de la aplicación o avanzar a la última escena. Además este menú contará con un texto que indique la utilidad de esta escena.\\

\begin{figure}[htb]
	\centering
	\includegraphics[width=0.5\textwidth]{./imagenes/echoMenu}
	\caption{Menú e instrucciones de la habitación}
\end{figure}

\quad El menú de desplegables se utiliza para para cambiar el tipo de material del que se componen las superficies de la habitación.\\

\begin{figure}[htb]
	\centering
	\includegraphics[width=0.3\textwidth]{./imagenes/materialMenu}
	\includegraphics[width=0.3\textwidth]{./imagenes/materialMenuDeploy}
	\caption{Menú para cambio de materiales}
\end{figure}
\FloatBarrier

\subsubsubsection{Componentes de EchoChamber}
	
\quad Los elementos no comunes de la escena EchoChamber determinarán el comportamiento de la misma:
\begin{itemize}
	\item ResonanceAudioRoom: delimita la zona cúbica donde el eco puede trabajar. Si se sale de esta delimitación el eco desaparece
	\item CubeRoom: en esta escena se tiene una escena cúbica que delimita el espacio de interacción
	\item Menu: compuestos por dos botones (uno para avanzar de escena y otro para salir de la aplicación) y un texto que explica cómo funciona la escena.
\item Icosahedron: es el punto de sonido de la escena, y al interaccionar con él se teletransportará a un lugar aleatorio de la escena para poder observar cómo esto afecta al eco de la habitación
\item Canvas: este canvas tiene dentro una serie de dropdown que interaccionan directamente con ResonanceAudioRoom para modificar los diferentes materiales que que componen sus caras.
\end{itemize}

\quad En particular ResonanceAudioRoom, Icosahedron y Canvasson especialmente importantes, pues son los que permiten trabajar con el eco en esta habitación, pero no adelantemos acontecimientos.\\

	\subsubsection{Forest}

\quad Esta escena pretende representar un pequeño bosque lleno de pájaros que van cantando, volando por las cercanías.\\

\quad Dispone de un menú para volver a la escena EchoChamber o salir de la aplicación.\\

\begin{figure}[htb]
	\centering
	\includegraphics[width=0.36\textwidth]{./imagenes/forestMenu}
	\includegraphics[width=0.38\textwidth]{./imagenes/forestMenuActive}
	\caption{Menú en el bosque}
\end{figure}

\quad En la parte superior del menú se encuentra un banner con los cinco pájaros dentro de la selección que se introduzcan en la aplicación. Cuando se visualice al tipo de pájaro objetivo, se marcará en ese banner.\\

\begin{figure}[htb]
	\centering
	\includegraphics[width=0.7\textwidth]{./imagenes/forestBirds}
	\caption{Pajaros en el bosque}
\end{figure}

\quad Debido a la carga que tendrá esta escena, es importante cambiar shaders y aplicar técnicas de optimización para evitar que la tasa de frames caiga demasiado.

\begin{figure}[htb]
	\centering
	\includegraphics[width=0.45\textwidth]{./imagenes/highShadders}
	\includegraphics[width=0.45\textwidth]{./imagenes/lowShadders}
	\caption{Comparativa entre shader de alta calidad y baja calidad}
\end{figure}
\FloatBarrier

\subsubsubsection{Componentes de Forest}

\quad Forest es con diferencia la escena con más carga de elementos de todo el proyecto, por lo que voy a diferenciar entre tipos de elementos.\\

\quad Primero debemos se va a hablar de los elementos que tiene que ver con la topografía de las escena:

\begin{itemize}
	\item Terrain: terreno irregular que crearemos utilizando las herramientas de Unity. Aprovechando el asset Fantasy Forest, le agregaremos proceduralmente hierba y árboles
	\item Rock: Utilizaremos los dos tipos de rocas del asset Hand Painted Forest para dar algo de personalidad a nuestro bosque
	\item Statue: utilizaremos los modelados de estatuas del asset anteriormente nombrado para darle más personalidad al bosque
\end{itemize}

\quad Ahora pasamos a ver los elementos asociados al comportamiento de los pájaros:

\begin{itemize}
	\item \_livingBirdsController: se encarga de lanzar los distintos tipos de pájaros para que interaccionen por el bosque
	\item lc\_perchTarget: hace de objetivo donde los pájaros aterrizan, siempre que este no esté en el suelo
	\item lb\_GroundTarget: determina un lugar de aterrizaje para los pájaros en el suelo
\end{itemize}

\quad Una aclaración importante es que lc\_perchTarget y lb\_GroundTarget afectan de manera distinta al árbol de animaciones de los pájaros.\\

\quad Forest posee, al igual que Init, un conjunto de muros invisibles llamado Wall que determina la zona por la que el jugador puede moverse.\\ 

\quad El último tipo de elemento que queda por remarcar es un menú con con las mismas interacciones que tenía el de EchoChamber, pero con una particularidad, posee un banner encima suyo que determina si hemos visto o no los pájaros en el postrado.\\


\newpage






	\thispagestyle{empty} 
	\textcolor[rgb]{1.00,1.00,1.00}{.} 
	\newpage %inserta un salto de página
	
	%--------------------------------------
	% IMPLEMENTACION
	%--------------------------------------
	
	\section{Implementación y control de versiones}

\subsection{Introducción}

\quad En esta parte se presentan las diferentes implementaciones necesarias, así como un control de versiones de la aplicación en la que se desmenuzan las diferentes versiones. \\

\quad Llevar un control de versiones es esencial como desarrollador, por lo que para este fin se utiliza la página \textit{github}, y como herramienta secundaria para la gestión del repositorio, la herramienta \textit{GitKraken}. Para el desarrollo de esta aplicación se ha seguido el paradigma de \textit{GitFlow} a la hora del control de versiones con el repositorio.\\

\subsection{Control de versiones}

\quad A continuación, se presenta el árbol de desarrollo de la aplicación en las que se ve un resumen de las modificaciones:

\begin{figure}[htb]
	\centering
	\includegraphics[width=0.8\textwidth]{./imagenes/git-tree1}
	\caption{Parte superior del gitTree}
\end{figure}
\FloatBarrier

\begin{figure}[htb]
	\centering
	\includegraphics[width=0.8\textwidth]{./imagenes/git-tree2}
	\caption{Parte inferior del gitTree}
\end{figure}
\FloatBarrier

\subsubsection{Cambios orgánicos en el planteamiento de la aplicación}

\quad El desarrollo de una aplicación no es ni mucho menos algo estático que sigue estrictamente los pasos definidos al inicio de éste durante la planificación. Un buen desarrollo debe saber adaptarse a requisitos que pudieron no tenerse en un principio.\\

\quad Este es el caso de la aplicación aquí presentada, de forma que me dispongo a enumerar unos cuantos cambios que surgieron a lo largo del desarrollo y que merecen ser resaltados por encima de los demás:

\begin{itemize}
	\item En la versión 5.0.0, desecho un control del movimiento del personaje basado en mando por un control basado únicamente en los que el cardboard nos presenta, pasando el botón de pantalla a haber que el personaje se mueva hacia adelante, y las interacciones con elementos de las escenas presentadas con un cargador de tiempo. Este cambio surge ante el planteamiento de  facilitar el uso para el usuario, al no depender de otro dispositivo externo, en este caso un mando.
	\item En la versión 6.0.0 la retícula de carga pasa a ser la propia retícula puntero que utiliza. Esto se realiza con la idea de evitar que la superposición de ambas retículas pueda marear al usaron cuando la de carga entra en escena.
	\item En la versión 6.0.0 se aplica \textit{Oclussion Culling} en la escena del bosque. Esto se hace para reducir su carga y así maximizar los frames de la escena.
\end{itemize}

\subsection{Implementación general: setUp de la aplicación}

\quad El trabajo de esta aplicación se ha desarrollado con Unity 2018.4.6f1. Esto se ha hecho así debido a que el hub de Unity determinaba que esta era la última versión estable de la aplicación en el momento del inicio de la programación de este proyecto.\\

\quad Deberemos configurar también la sdk para que Unity pueda trabajar. Normalmente si ya se tiene la sdk de android instalada, Unity la reconocerá sin necesidad de añadirla manualmente.\\

\quad Debemos tener en cuenta que, en "Edit>Project Settings", debemos activar la casilla \textit{Virtual Reality Supported} de XR Settings para android en el "Player", y fijar en la pestaña "Other Settings" el \textit{Minimum API Leve} a nivel 19.\\

\subsubsection{Paquetes a descargar y assets gratuitos}

\quad Los paquetes que necesitaremos para poder utilizar la tecnología necesaria en Unity son:

\begin{itemize}
	\item ResonaceAudioForUnity\_<versión\_deseada>.unitypackage \cite{DResonance}
	\item GoogleVRForUnity\_<versión\_deseada>.unitypackage \cite{DGvr}
\end{itemize}

\quad El paquete GoogleVR es un set de utilidades desarrolladas por google para facilitar el desarrollo de una aplicación con realidad virtual que además tiene versión para varios motores.\\

\quad El paquete Resonance es un set de herramientas con los algoritmos pertinentes para poder simular los distintos aspectos de un sonido envolvente, con distintos efecto, percepción de profundidad (conocido como sonido 3D) y sonido 8D.\\

\quad En la documentación hay un enlace para poder acceder a todas las características aplicadas a Unity para ambos paquetes \cite{Gvr} \cite{Resonance}.\\


\quad Para instalar un paquete descargado, solo habrá que introducir el paquete manualmente desde la pestaña \textit{Assets>Import Package>Custom Package}.\\

\quad Por otra parte, los assests gratuitos sacados de la store de Unity son los siguiente:

\begin{itemize}
	\item Fantasy Forest
	\item Hand Painted Forest Lite
	\item Living Birds
\end{itemize}

\quad \textit{Fantasy Forest} ha sido utilizado para extraer el modelaje de los árboles y la textura 2D para la hierba. La aplicación de los árboles en un terreno, así como de la hierba se hace siguiendo el procedimiento standard de Unity.

\begin{figure}[htb]
	\centering
	\includegraphics[width=0.7\textwidth]{./imagenes/fantasyForest}
	\caption{Fantasy Forest}
\end{figure}
\FloatBarrier

\quad \textit{ Hand Painted Forest Lite} ha sido descargado exclusivamente por el modelado de las estatuas, las cuáles han sido integradas al terreno en el que el jugador se sitúa.

\begin{figure}[htb]
	\centering
	\includegraphics[width=0.7\textwidth]{./imagenes/handPaintedForest}
	\caption{Hand Painted Forest Lite}
\end{figure}
\FloatBarrier

\quad \textit{Living Birds} posee los modelados y animaciones de los pájaros, además de un sistema para hacer que los pájaros se generarán automáticamente para simular un entorno orgánico. en el momento en el que se utilizó este asset, lo único implementado fueron los modelajes con sus animaciones, ya que la mayoría de script asociados a la generación automática estaban vacíos, o directamente no existían, por lo que parte del desarrollo se dedicó a rehacer gran parte de este asset.

\begin{figure}[htb]
	\centering
	\includegraphics[width=0.7\textwidth]{./imagenes/livingBirds}
	\caption{Living Birds}
\end{figure}
\FloatBarrier

\subsection{Jugador}

\quad El jugador estará formado por un GameObject llamado \textit{RealPlayer} que contendrá la "MainCamera". Se introducirá como hijo de ésta el prefab \textit{GvrReticlePointer} que pondrá en el jugador la retícula que utilizaremos para las interacciones. Hay una serie de cambios en la retícula que se explican en el siguiente subapartado.\\

\begin{figure}[htb]
	\centering
	\includegraphics[width=0.5\textwidth]{./imagenes/player}
	\caption{Aspecto del Jugador en el árbol de escena}
\end{figure}
\FloatBarrier

\subsubsection{Controles de movimiento}

\quad Se ha implementado para el movimiento que cuando se hace click en la pantalla, el jugador se mueve. Esto se consigue añadiendo un script de C\# a \textit{RealPlayer} que contendrá el siguiente código:

\lstset{language=[sharp]C, breaklines=true, basicstyle=\footnotesize}
\begin{lstlisting}[frame=single, caption={PlayerWalk.cs}]
using System.Collections;
using System.Collections.Generic;
using UnityEngine;

public class PlayerWalk : MonoBehaviour
{
    public int playerSpeed;

    // Update is called once per frame
    void Update()
    {
        if (Input.GetButton("Fire1")) {
            transform.position = transform.position + Camera.main.transform.forward * playerSpeed * Time.deltaTime;
        }
    }
}

\end{lstlisting}

\subsubsection{Control de cámara en VR}

\quad Simplemente se añadirá al árbol de escena el prefab de GoogleVR \textit{GvrEditorEmulator}, que en ordenador proporciona una vista normal mientras que al instalar la apk en un dispositivo, este mostrará la pantalla dividida y adaptada para ser utilizada en un dispositivo de visionado como las \textit{cardboard}.\\

\quad El control simulado en el PC se utiliza de la siguiente manera:
\begin{itemize}
	\item \textit{Alt + move} mouse para rotar sobre uno mismo y cambiar la dirección a la que se orienta el jugador
	\item \textit{Ctrl + move} mouse para rotar sobre el eje que sale desde el jugador hacia adelante
\end{itemize}

\quad No voy a entrar en detalles con el código del objeto en cuestión ya que no es código propio.\\

\subsubsection{Retícula}

\quad La retícula que viene por defecto en unity no será necesaria en el caso presentado, pues es mejor y más eficiente utilizar el script \textit{GvrPointerPhysicsRaycaster.cs}, adjunto con el paquete de GoogleVR.\\

\quad La retícula es el medio para interactuar con el medio, pero no interesa que se active automáticamente, si no que será preciso que haya un tiempo de espera para que el usuario decida si se va a llevar a cabo esta interacción o si por el contrario desea cancelarla.\\

\quad Con motivo de ello, se aprovechará la capacidad de la retícula de google de ampliarse en una interacción (efecto producido con el prefab \textit{GvrEventSystem}.) y se modificará el ángulo mostrado en pantalla de ésta el cual se irá modificando, convirtiéndola así en una barra de carga circular. Para esto se requieren las siguientes modificaciones en el shader \cite{ShadersTut} de Google:

\begin{itemize}
	\item Añadir la propiedad Angle
	\item Añadir el float que la representará
	\item Sustituir la función \textit{vert} para para que tenga en cuenta esta nueva propiedad en el shader
\end{itemize}
\FloatBarrier

\quad A continuación se adjunta el resultado final para ver como debe quedar el código:

\lstset{language=[sharp]C, breaklines=true, basicstyle=\footnotesize}
\begin{lstlisting}[frame=single, caption={GvrReticleShader.shader}]
// Copyright 2015 Google Inc. All rights reserved.
//
// Licensed under the Apache License, Version 2.0 (the "License");
// you may not use this file except in compliance with the License.
// You may obtain a copy of the License at
//
// http://www.apache.org/licenses/LICENSE-2.0
//
// Unless required by applicable law or agreed to in writing, software
// distributed under the License is distributed on an "AS IS" BASIS,
// WITHOUT WARRANTIES OR CONDITIONS OF ANY KIND, either express or implied.
// See the License for the specific language governing permissions and
// limitations under the License.

// Trololo

Shader "GoogleVR/Reticle" {
Properties {
_Color ("Color", Color) = ( 1, 1, 1, 1 )
_InnerDiameter ("InnerDiameter", Range(0, 10.0)) = 1.5
_Angle("Angle", Range(0, 360)) = 180 // Propiedad con la que trabajaremos en el shader
_OuterDiameter ("OuterDiameter", Range(0.00872665, 10.0)) = 2.0
_DistanceInMeters ("DistanceInMeters", Range(0.0, 100.0)) = 2.0
}

SubShader {
Tags { "Queue"="Overlay" "IgnoreProjector"="True" "RenderType"="Transparent" }
Pass {
Blend SrcAlpha OneMinusSrcAlpha, OneMinusDstAlpha One
AlphaTest Off
Cull Back
Lighting Off
ZWrite Off
ZTest Always

Fog { Mode Off }
CGPROGRAM

#pragma vertex vert
#pragma fragment frag

#include "UnityCG.cginc"

uniform float4 _Color;
uniform float _InnerDiameter;
uniform float _OuterDiameter;
uniform float _DistanceInMeters;
uniform float _Angle; // Variable con la que trabajar 

struct vertexInput {
float4 vertex : POSITION;
};

struct fragmentInput{
float4 position : SV_POSITION;
};

/* fragmentInput vert(vertexInput i) {
float scale = lerp(_OuterDiameter, _InnerDiameter, i.vertex.z);

float3 vert_out = float3(i.vertex.x * scale, i.vertex.y * scale, _DistanceInMeters);

fragmentInput o;
o.position = UnityObjectToClipPos (vert_out);
return o;
}
*/


// Nueva version del metodo vert 
// que tiene en cuenta elangulo a la hora del dibujado
fragmentInput vert(vertexInput i) {
	 fragmentInput o;
	 if (_DistanceInMeters < 20) {
		 // 180* = 2.9
		 // limit = - (a - 180) * 2.9/180
		 float limit = -(_Angle - 180) * 0.0161111111;
		 float3 vert_out = float3(0, 0, _DistanceInMeters);
		 float a = -atan2(i.vertex.x, i.vertex.y);
		 if (a >= limit) {
			 float scale = lerp(_OuterDiameter, _InnerDiameter, i.vertex.z);
			 vert_out = float3(i.vertex.x * scale, i.vertex.y * scale, _DistanceInMeters);
		 }
		 o.position = UnityObjectToClipPos(vert_out);
	 }
	 return o;
}

fixed4 frag(fragmentInput i) : SV_Target {
fixed4 ret = _Color;
return ret;
}

ENDCG
}
}
}

\end{lstlisting}

\subsubsubsection{Interacción de la retícula}

\quad Ahora entra en juego como conseguir la interacción con el objeto. Para ello se requieren dos pasos muy importantes: \\

\begin{itemize}
	\item Añadir al objeto un \textit{EventTrigger} que trabajará cuando el puntero entre dentro del área que ocupa el objeto y cuando salga.
	\item Añadir un script al objeto que en este caso se ha llamado \textit{GVRButton.cs}.
\end{itemize}
\FloatBarrier

\quad Lo primero debe hacerse para determinar las funciones GvrOn y GvrOff que determinarán el comportamiento ante la entrada y la salida del objeto. El segundo determina la acción que se llevará a cabo a partir de que termine el evento de entrada, que en este caso que la barra de carga termine de llenarse.\\

\quad Se adjuntan seguidamente es aspecto de un objeto que posee estas cualidades y el código asociado a \textit{GVRButton.cs}.\\

\lstset{language=[sharp]C, breaklines=true, basicstyle=\footnotesize}
\begin{lstlisting}[frame=single, caption={GVRButton.cs}]
using System.Collections;
using System.Collections.Generic;
using UnityEngine;
using UnityEngine.UI;
using UnityEngine.Events;

public class GVRButton : MonoBehaviour
{
    GameObject pointer;
    public UnityEvent GVRClick;
    public float totalTime = 2;
    bool gvrStatus;
    public float gvrTimer;
    Renderer rend;
    float angle;
    bool counted;

    private void Start()
    {
        pointer = GameObject.FindWithTag("gvrpointer"); 	// Busqueda del puntero con elq ue se va a trabajar
        rend = pointer.GetComponent<Renderer>();		// Extraccion del renderer del puntero
        counted = false;						// Filtro para el conteo de parajaros en el bosque
    }

    // Update is called once per frame
    void Update()
    {
        if (gvrStatus)
        {
            gvrTimer += Time.deltaTime;			// Calculo de la fraccion de tiempo de la carga en la que se esta trabajando
            angle = gvrTimer / totalTime * 360;		// Calculo de la fraccion de arco a dibujar
            rend.material.SetFloat("_Angle", angle);	// Cambiar la variable angulo de la reticula
        }

        if (gvrTimer > totalTime)
        {
            GVRClick.Invoke();					// Efectuar un click
            gvrStatus = false;					// Determinar que se ha terminado la ejecucion
	 GvrOff();
        }
    }

// Metodo de inicio de interaccion de la reticula 
    public void GvrOn()
    {
        gvrStatus = true;
    }

// Metodo de finalizacion de interaccion de la reticula
    public void GvrOff()
    {
        gvrStatus = false;
        gvrTimer = 0;
        rend.material.SetFloat("_Angle", 360f);
    }

// Metodo para marcar los pajaros vistos en el cartel del bosque
    public void checkBird(string bird)
    {
        if (!counted)
        {
            GameObject.FindWithTag(bird).GetComponent<Text>().text = "OK";
            counted = true;
        }
    }
}

\end{lstlisting}

\begin{figure}[htb]
	\centering
	\includegraphics[width=0.8\textwidth]{./imagenes/cube}
	\caption{Cubo con interacciones}
\end{figure}
\FloatBarrier

\quad En el código se muestra como acceder al shader y modificar el ángulo que creamos en el apartado anterior dentro de la función update.\\

\subsubsection{Aplicar sonido 8D a un objeto}

\quad Esta parte se soluciona rápidamente añadiendo los dos siguientes elementos al objeto emisor de sonido y a la mainCamera:\\

\begin{itemize}
	\item El script \textit{ResonanceAudioListener.cs} a la main camera de la escena.
	\item El prefab\textit{ResonanceAudioSource} al objeto que hará de emisor.
\end{itemize}

\quad Es importante añadir el audio que queremos que se reproduzca en la propiedad \textit{Audio Source} para que los scripts puedan trabajar.

\begin{figure}[htb]
	\centering
	\includegraphics[width=0.6\textwidth]{./imagenes/audiosource}
	\includegraphics[width=0.6\textwidth]{./imagenes/audiolistener}
	\caption{Script ResonaceAudioSource y ResonanceAudioListener en el Inspector}
\end{figure}
\FloatBarrier

\subsection{Implementación por escenas}

\quad Ahora analizaremos los detalles más importantes implementados por escena.\\
	\subsubsection{Escena Init}
		\subsubsubsection{Cubo para interactuar}

\quad Ya se ha hablado de el cubo de la primera escena que interactúa con el usuario cuando lo mira, pero tiene varias acciones que realizar en el momento de la interacción:

\begin{itemize}
	\item Cambiar el audio que se está reproduciendo durante la ejecución.
	\item Cambiar la escena en la que se encuentra el jugador.
	\item Cambiar el material del cubo para que se haga visible.
\end{itemize}

\quad La primera acción requiere que por nuestra parte creemos un script que active la subrutina que cambie en ejecución el audio reproducido. La segunda acción requiere que cuando se termine de reproducir el nuevo audio se cargue la siguiente escena. El siguiente código muestra cómo hacer ambas cosas desde la subrutina WaitFinish:\\

\lstset{language=[sharp]C, breaklines=true, basicstyle=\footnotesize}
\begin{lstlisting}[frame=single, caption={ChangeSound.cs}]
using System.Collections;
using System.Collections.Generic;
using UnityEngine;
using UnityEngine.SceneManagement;
using UnityEngine.Timeline;

public class ChangeSound : MonoBehaviour
{
    AudioSource myaudio;
    Material mymaterial;
    Renderer rend;

    public void changeAudio()
    {
        mymaterial = Resources.Load<Material>("RedMat");	// Busca el material por el que se va a cambiar
        rend = GetComponentInParent<Renderer>();		// Se obtiene el renderer del padre
        rend.enabled = true;						// Nos aseguramos de que este activo
        rend.sharedMaterial = mymaterial;				// Se cambia el material
        StartCoroutine(WaitFinish());				// Se llama a la subrutina que cambia el audio durante la ejecucion
    }


// Subrutina que espera a que un audio termine de reproducirse completamente para avanzar a la siguiente escena 
    IEnumerator WaitFinish()
    {
        myaudio = GetComponent<AudioSource>();			// Localiza el componente AudioSource de la fuente de sonido
        myaudio.clip = Resources.Load<AudioClip>("porfinteveo3");	// Cambia el audio reproducido por el foco de sonido
        myaudio.Play();							// Reproduce el nuevo audio
        yield return new WaitForSeconds(myaudio.clip.length);	// Espera la longitud del audio reproducido
        SceneManager.LoadScene("EcoChamber");			// Carga de la nueva escena
    }
}

\end{lstlisting}

\quad En el mismo código se incluye dentro de la función changeAudio el método para poder cambiar el material de un GameObject.\\

	\subsubsection{Escena Echo Chamber}
		\subsubsubsection{Aplicar eco}
\quad Dentro del GameObject que define la habitación se debe añadir el prefab ResonanceAudioRoom, modificando el tamaño de este para que se ajuste al tamaño de la habitación.

\begin{figure}[htb]
	\centering
	\includegraphics[width=0.55\textwidth]{./imagenes/echoroom}
	\caption{ResonanceAudioRoom ajustado a la habitación de la escena}
\end{figure}
\FloatBarrier

\quad Desde el script asociado al prefab, se pueden modificar los materiales que componen las diferentes caras del cubo que delimita el espacio que dispondrá de eco, así como las características del eco en cuestión (reflectividad, tiempo, ganancia en DB, brillo).

\begin{figure}[htb]
	\centering
	\includegraphics[width=0.8\textwidth]{./imagenes/audioroom}
	\caption{Script ResonaceAudioRoom}
\end{figure}
\FloatBarrier

		\subsubsubsection{Cambiar en escena los materiales de la habitación}

\quad En la aplicación se encuentran varios materiales implementados para los diferentes lados del cubo que delimita la zona. Para simplificarlo, en esta aplicación solo se han implementado una pequeña porción para demostrar un ejemplo de cambio de material dinámico de estos.\\

\quad La implementación de \textit{dropdown} ha sido llevada a cabo mediante el objeto suministrado por Unity. Debido a que no se dispone de una versión de pago para el desarrollo de esta aplicación, no se pueden tocar internamente el script que lo pone en funcionamiento, pero hay total libertad a la hora de añadir lo elementos que se requieran en la implementación.\\

\quad Siguiendo ese razonamiento, durante este desarrollo se han añadido un \textit{eventTrigger} y el script anteriormente mencionado \textit{GVRButton}, haciendo la configuración requerida para poder interactuar con los elementos internos del dropdown.\\

\begin{figure}[htb]
	\centering
	\includegraphics[width=0.5\textwidth]{./imagenes/dropdown}
	\caption{Dropdown en la jerarquía de escena}
\end{figure}
\FloatBarrier

\begin{figure}[htb]
	\centering
	\includegraphics[width=0.8\textwidth]{./imagenes/dropDownScriptUnity}
	\caption{Script de un dropdown}
\end{figure}

\quad Como se puede apreciar, no hay problema a la hora de definir los elementos del desplegable, a los que se les asignará por defecto un valor entero en el orden en el que son añadidos.\\

\quad Para poder acceder al valor asignado, en la zona de \textit{OnValueChage} se ha añadido al elemento \textit{ResonanceAudioRoom} como GameObject, y se llama a una de las funciones que se montaran dentro de su propio código llamadas \textit{changeMaterialRight}, \textit{changeMaterialLeft}, \textit{changeMaterialFront}, \textit{changeMaterialBack}, \textit{changeMaterialCeiling} y \textit{changeMaterialFloor}.\\

\begin{figure}[htb]
	\centering
	\includegraphics[width=0.8\textwidth]{./imagenes/materialMenu}
	\includegraphics[width=0.8\textwidth]{./imagenes/materialMenuDeploy}
	\caption{Menú de cambio de material}
\end{figure}
\FloatBarrier

\quad La implementación de las seis funciones mencionadas es prácticamente la misma, ya que solo se varía la referencia al lado sobre el que se vana hacer las modificaciones.\\

\lstset{language=[sharp]C, breaklines=true, basicstyle=\footnotesize}
\begin{lstlisting}[frame=single, caption={Función changeMaterialLeft}]
...
public ResonanceAudioRoomManager.SurfaceMaterial rightWall =
	ResonanceAudioRoomManager.SurfaceMaterial.ConcreteBlockCoarse;
...
public void changeMaterialLeft(Dropdown mine) {
        int value = mine.value;	// Se obtiene el valor que tiene la opcion seleccionada del dropdown

        switch (value)			// Dependiendo del valor obtenido se el material con el que se cambiara 
        {
            case 0:
                leftWall = ResonanceAudioRoomManager.SurfaceMaterial.ConcreteBlockCoarse;
                break;
            case 1:
                leftWall = ResonanceAudioRoomManager.SurfaceMaterial.Transparent;
                break;
            case 2:
                leftWall = ResonanceAudioRoomManager.SurfaceMaterial.CurtainHeavy;
                break;
            case 3:
                leftWall = ResonanceAudioRoomManager.SurfaceMaterial.GlassThick;
                break;
            case 4:
                leftWall = ResonanceAudioRoomManager.SurfaceMaterial.Marble;
                break;
            case 5:
                leftWall = ResonanceAudioRoomManager.SurfaceMaterial.Metal;
                break;
            case 6:
                leftWall = ResonanceAudioRoomManager.SurfaceMaterial.PolishedConcreteOrTile;
                break;
            default:
                break;
        }
    }

\end{lstlisting}


		\subsubsubsection{Icosaedro el sonido}

\quad Respecto al icosaedro solo queda añadir la función que lo hace teleportarse cuando el evento entra en acción. Esta función se recoge en el siguiente código:\\

\lstset{language=[sharp]C, breaklines=true, basicstyle=\footnotesize}
\begin{lstlisting}[frame=single, caption={Teleporter.cs}]
using System.Collections;
using System.Collections.Generic;
using UnityEngine;

public class Teleporter : MonoBehaviour
{
    Vector3 destination; // Almacenara la posicion random
    RectTransform rt;
    
// Se cambia la posicion del icosaedro por una posicion random
    public void randomPlace()
    {
        destination = new Vector3(Random.Range(-8, 8), Random.Range(1, 8), Random.Range(-8, 8));
        rt = GetComponent<RectTransform>();
        rt.transform.localPosition = destination;
    }
}

\end{lstlisting}

	\subsubsection{Escena Forest}
		\subsubsubsection{Oclusion Culling}
\quad Occlusion Culling \cite{Culling} es una característica que desactiva el renderizado de objetos cuando actualmente no estén visibles por la cámara puesto que están oscurecidos (occluded) por otros objetos. Esto no sucede automáticamente en gráficas computacionales 3D ya que la mayoría de veces los objetos que están más lejos de la cámara son dibujados primero y los objetos más cercanos son dibujados encima de estos (esto se llama “overdraW”). El Occlusion Culling es diferente del Frustum Culling, ya que este solamente desactiva los renderers para objetos que están fuera del área visible de la cámara, pero no desactiva nada oculto de la vista por overdraw.\\

\quad Para que el culling funcione, los objetos que se deseen ver afectados por él deben tener la casilla Static activada. De esta forma evitamos que objetos que se mueven se vean afectados por la desaparición en el dibujado.\\

\begin{figure}[htb]
	\centering
	\includegraphics[width=0.8\textwidth]{./imagenes/cullingdata}
	\caption{Parámetros aplicados en la aplicación para el culling}
\end{figure}
\FloatBarrier

\begin{figure}[htb]
	\centering
	\includegraphics[width=0.40\textwidth]{./imagenes/culling1}
	\includegraphics[width=0.40\textwidth]{./imagenes/culling2}
	\caption{Script ResonaceAudioRoom}
\end{figure}
\FloatBarrier

		\subsubsubsection{Pájaros y su búsqueda}

\quad El sonido en los pájaros se genera de la misma forma que en cualquier elemento con sonido (el cubo o el icosaedro), aunque presentan una pequeña diferencia. Donde encontremos dentro del script \textit{lb\_Bird.cs} una reproducción de uno de los cuatro audios que componen los sonidos del pájaro, se debe tener en cuenta que ahora el pájaro no contará con un AudioSource propio, si no que ese componente se encontrará dentro del objeto hijo ResonanceAudioSource que se le añadirá.\\

\quad Para poder acceder al componente de este objeto hijo, se debera cambiar la linea que hace el play por lo siguiente:\\

\lstset{language=[sharp]C, breaklines=true, basicstyle=\footnotesize}
\begin{lstlisting}[frame=single, caption={Ejemplo de cambio de audio para pájaro}]
	this.transform.Find("ResonanceAudioSource").gameObject.GetComponent<AudioSource>().PlayOneShot (song1,1);
\end{lstlisting}

\quad Deben hacerse cuatro cambios en el script \textit{lb\_Bird.cs}.\\

\quad Lo último a tener en cuenta en esta escena, es cambiar el shader de los materiales que componen la escena a \textit{Movile/Diffuse}. De esta forma se garantiza que el shader está optimizado para trabajar en un móvil.\\

\quad Ahora toca la implementación para poder interactuar con los distintos pájaros que aparecen en la escena, de forma que cuando interactuemos con dentro de la lista \textit{Most Wanted!!} situada encima del menú en el bosque, ésta chequee a ese ave en particular y la marque con un \textit{OK}.\\

\quad Con ese fin se ha escrito dentro del script GVRButton la función checkBird, a la cual se le pasa un string, el cual será el tag del texto que deberemos cambiar de vacío a "OK".\\

\begin{lstlisting}[frame=single, caption={función checkBird}]
// Metodo para marcar los pajaros vistos en el cartel del bosque
    public void checkBird(string bird)
    {
        if (!counted)
        {
            GameObject.FindWithTag(bird).GetComponent<Text>().text = "OK";
            counted = true;
        }
    }
\end{lstlisting}

\quad Es importante recordar que hay que modificar el prefab de cada pájaro que se quiera convertir en interactivo mediante los pasos que ya se explicaron con anterioridad.\\


\newpage






	\thispagestyle{empty} 
	\textcolor[rgb]{1.00,1.00,1.00}{.} 
	\newpage %inserta un salto de página
	
	%--------------------------------------
	% PRUEBAS
	%--------------------------------------
	
	\section{Pruebas, testeos y fallos}

\subsection{Introducción}

\quad Como es evidente, a lo largo de un desarrollo, por pequeño que sea, aparecen problemas que el desarrollador debe afrontar utilizando sus dotes deductivas y de búsqueda. Dicho esto, es tarea del mismo buscar una solución que arregle el problema encontrado, o una solución alternativa que cambie la forma de enfocar el objetivo.\\

\quad A continuación se presentan un par de casos que aparecieron durante este desarrollo.\\

\subsection{El caso de las retículas rebeldes}

\quad Siguiendo un tutorial de youtube, se intento hacer un temporizador para la interacciones de la retícula con los objetos.\\

\quad El resultado era satisfactorio dentro del entorno en el PC, pero los problemas comenzron al pasar la aplicación al dispositivo móvil, ya que la reticula de carga no se centraba correctamente, provocando una sensación de mareo al no saber sobre que reticula centrarse.\\ 

\quad La solucion por la que se optó al descubrir que todo era producido por un bug interno producido por incompatibilidades entre la versión de Unity y la del paquete GoogleVR no fue igualar versiones, ya que hubiera implicado rehacer practicamente todo el proyecto, si no que se opto por transformar la retícula base y convertirla en un temporizador cuando fuese necesario. Esto implico, como se explica en el apartado \textit{6.5.3. Retícula}, realizar una serie de modificaciones en el shader de google para el material de la retícula.\\

\subsection{La resina del bosque no me permite andar}

\quad La gran carga gráfica del bosque no solo lagueaba la aplicación, si no que además inutilizaba la retícula y el movimiento del personaje, mientras que el movimiento de cámara seguía en funcionamiento.\\

\quad La solución paso por reducir el detalle de los modelados, pero esto no termino de solucionar el problema, por lo que se optó por el culling, asi como cambiar todo los shaders a a los menos pesados con los que contaba Unity.\\

\quad Esta solución ha hecho que el bosque sea jugable por fin, pero si se nota que va un poco más lento que las otras dos escenas, cuya carga poligonal el mucho menor.\\  

\subsection{Biclope involuntario}

\quad Este problema es algo que ocurre a todo los desarrolladores, pero es interesante comentarlo para que quede constancia de que el error humano existe y a veces es necesario descansar y continuar con el trabajo más adelante.\\

\quad El problema que se presentó fue que en algunas escenas, la retícula no interaccionaba con los elementos dispuestos para ello. Esto fue un problema serio en su momento, pues parecía que tocaría rehacer gran parte del trabajo, pero después de un descanso, se percibió que la retícula estaba duplicada en la escena, lo que probocaba que no interaccionara correctamente. Una vez eliminada la sobrante, todo funcionó como la seda.\\

\subsection{Mi código, mis normas}

\quad Aqui nos encontramos con el problema de la encapsulación del código, que ha probocado que las iteracciones con los \textit{dropdown} se vuelvan más problemáticas.\\

\quad Debido a que no se puede entrar en el script del dropdown para poder hacer modificaciones, no se puede acceder a la lista de elementos de forma sencilla, y mucho menos modificar la interacción con estos elementos, de forma que el click de andar y el de presionar sobre esos elementos se solapa en lugar de poder hacer la interacción con un temporizador y la retícula (como en los otros casos).\\

\quad La solución esta siendo montar un dropdown personal, pero ante la falta de tiempo se ha optado por presentar esta versión con este pequeño bug, que realmente no entropece la experiencia con el sonido, que es lo que se pretende condseguir.\\



\newpage






	\thispagestyle{empty} 
	\textcolor[rgb]{1.00,1.00,1.00}{.} 
	\newpage %inserta un salto de página
	
	%--------------------------------------
	% ANALISIS DE NEGOCIO
	%--------------------------------------
	
	%\input{capitulos/08_Analisis_negocio}
	%\thispagestyle{empty} 
	%\textcolor[rgb]{1.00,1.00,1.00}{.} 
	%\newpage %inserta un salto de página
	
	%--------------------------------------
	% CONCLUSIONES Y FUTUROS
	%--------------------------------------
	
	%\section{Conclusiones}

\quad Esta aplicación a resultado apasionate, pues realmente creo que me ha hecho crecer como desarrollador. Digo esto, porque trabajar en un proyecto tuyo, que sabes que quieres hacer y que tu mismo propusiste, puede ser abrumador, pero acaba sacando lo mejor de ti.\\

\quad Por la falta de potencia de procesamiento, tanto del equipo de desarrollo como del movil con el que se iba a trabajar, ubo una idea uqe se tuvo que desechar, que era aplicar Raytracing al concepto de 8D. Esto se va a desarrollar ahora mismo en el apartado siguiente.\\

\subsection{Lineas futuras, Ray tracing}

\subsubsection{Definición y planteamiento para sonido}

\quad El raytracing es un algoritmo para síntesis de imágenes tridimensionales propuesto inicialmente por Turner Whitted en 1980. Está basado en el algoritmo de determinación de superficies visibles de Arthur Appel denominado Ray Casting (1968), en el que se se determinan las superficies visibles en la escena que se quiere sintetizar trazando rayos desde el observador (cámara) hasta la escena a través del plano de la imagen. Se calculan las intersecciones del rayo con los diferentes objetos de la escena y aquella intersección que esté más cerca del observador determina cuál es el objeto visible.\\

\quad En el Raytracing se extiende la idea de trazar los rayos para determinar las superficies visibles con un proceso de sombreado (cálculo de la intensidad del píxel) que tiene en cuenta efectos globales de iluminación como pueden ser reflexiones, refracciones o sombras arrojadas. Para simular los efectos de reflexión y refracción se trazan rayos recursivamente desde el punto de intersección que se está sombreando dependiendo de las características del material del objeto intersecado. Para simular las sombras arrojadas se lanzan rayos desde el punto de intersección hasta las fuentes de luz. Estos rayos se conocen con el nombre de rayos de sombra (shadow rays).\\

\quad Se puede apreciar que en los parrafos anteriores solo se habla de gráficos, lo cuál plantea cómo aplicar esto a sonido. En realidad el concepto es muy similar al lo anterior, se imiten los rayos, pero determinamos las variaciones del sonido por los rebotes en los diferentes elementos en una habitación, determinando el ángulo de entrada más optimo para ese fin.\\

\begin{figure}[htb]
	\centering
	\includegraphics[width=0.5\textwidth]{./imagenes/pizarra}
	\caption{Esquema conceptual en pizarra del raytracing}
\end{figure} 

\quad Gracias a la industria del videojuego, no estamos lejos de poder disfrutar del raytracing de una forma más común, ya que juegos como Cyberpunk 2077 (a día de la redacción de este documento, el juego no ha salido al mercado) plantean el uso de esta tecnología, pero los equipos de los que disponen los jugadores aún no estan a la altura de una carga computacional tan grande. Con todo, solo se puede decir que el cielo es el límite, y que en un futuro no tan lejano, todo lo plateado sobre esta tenología que suena a ciencia ficción, disfrutaremos de todo esto como si fuera lo más común y normal.\\ 

\newpage



	%\thispagestyle{empty} 
	%\textcolor[rgb]{1.00,1.00,1.00}{.} 
	%\newpage %inserta un salto de página
	
	%-----------------------------------------------------------------------
	%							     APENDICES
	%-----------------------------------------------------------------------
	\appendix
	
	%--------------------------------------
	% MANUAL DE USUARIO
	%--------------------------------------
	
	\section{Manual de Usuario}

\subsection{Material necesario}

\quad A parte de la aplicación descargada, el usuario debe disponer de los siguientes elementos:
\begin{itemize}
	\item Una cardboard o similar
	\item Unos cascos estéreo
	\item Un móvil con sistema mínimo \textit{android 4.4 Kit Kat}
\end{itemize}

\subsection{Acciones del jugador}

\quad Las acciones que el usuario puede desarrollar son las siguientes;

\begin{itemize}
	\item Andar hacia adelante: presionar el botón de la cardboard.
	\item Girar la cámara: girar sobre uno mismo para que el acelerómetro determine hacia dónde miras.
	\item Interaccionar con un elemento: si se puede interaccionar, al apuntar la retícula hacia el esta cambiará a una barra de carga que, al terminar, desata la interacción asociada.
\end{itemize}

\subsection{Objetivos en las pantallas}

\subsubsection{Init}

\quad Esta es la primera escena de la aplicación y está pensada para acostumbrarse a la interfaz de usuario.\\

\quad La tarea es bien sencilla en este caso: encontrar al ser invisible que nos está llamando desde algún lugar.\\ 

\quad Cuando por fin lo encontremos, nos mandará a otra escena después de despedirse.\\

\subsubsection{EchoChamber}

\quad Aparecemos en una habitación cúbica después de despedirnos del amigo invisible donde suena una tonadilla alegre.\\

\quad En esta zona hay tres particularidades:
\begin{itemize}
	\item Un icosaedro
	\item Un menú para cerrar la aplicación o avanzar a la siguiente zona
	\item Un menú que indica cambiar el material  
\end{itemize}

\quad ¿No hay mucho eco aquí? El usuario deberá interaccionar con los distintos elementos y comprobar qué efectos tienen sus acciones en ese entorno.  

\subsubsection{EchoChamber}

\quad Esta ya es la última zona, un pequeño bosque con distintas aves.Prueba a acercarte e interaccionar con algunas, quizás interaccionar con ellas modifique el cartel en el que se lee “Most Wanted!!”.\\

\newpage





	\thispagestyle{empty} 
	\textcolor[rgb]{1.00,1.00,1.00}{.} 
	\newpage

	%--------------------------------------
	% ANECDOTAS	
	%--------------------------------------
	
	\section{Anecdotas}

\subsection{Biclope involuntario}

\quad Este problema es algo que ocurre a todo los desarrolladores, pero es interesante comentarlo para que quede constancia de que el error humano existe y a veces es necesario descansar y continuar con el trabajo más adelante.\\

\quad El problema que se presentó fue que en algunas escenas, la retícula no interaccionaba con los elementos dispuestos para ello. Esto fue un problema serio en su momento, pues parecía que tocaría rehacer gran parte del trabajo, pero después de un descanso, se percibió que la retícula estaba duplicada en la escena, lo que provocaba que no interaccionasen correctamente. Una vez eliminada la sobrante, todo funcionó como la seda.\\

\newpage
	\thispagestyle{empty} 
	\textcolor[rgb]{1.00,1.00,1.00}{.} 
	\newpage

	%--------------------------------------
	% GITFLOW	
	%--------------------------------------
	
	\section{GitFlow}
\subsection{¿Qué es GitFlow?}

\quad Gitflow es un diseño de flujo de trabajo Git que se publicó por primera vez y se hizo popular por \textit{Vincent Driessen} en \textit{nvie}. El flujo de trabajo de Gitflow define un modelo de ramificación estricto diseñado en torno al lanzamiento del proyecto. Esto proporciona un marco robusto para gestionar proyectos más grandes.\\

\quad Gitflow es ideal para proyectos que tienen un ciclo de lanzamiento programado, pues este flujo de trabajo no agrega nuevos conceptos o comandos más allá de lo que se requiere para el Flujo de trabajo de la rama de funciones, si no que asigna roles muy específicos a diferentes ramas y define cómo y cuándo deben interactuar. Además de las ramas de características, utiliza ramas individuales para preparar, mantener y grabar lanzamientos. Por supuesto, también puede aprovechar todos los beneficios del flujo de trabajo de Branch Branch: solicitudes de extracción, experimentos aislados y una colaboración más eficiente.\\ 

\subsection{¿Cómo funciona? \cite{GitKraken2} \cite{GitFlow}}

\subsubsection{Ramas Develop \& Master}

\quad En lugar de una sola rama, este flujo de trabajo usa dos ramas para registrar el historial del proyecto.\\ 

\quad La rama \textit{master} almacena el historial de lanzamiento oficial, y la rama \textit{develop} sirve como una rama de integración de características. También es conveniente etiquetar todos los commits en master con un número de versión.\\

\begin{figure}[htb]
	\centering
	\includegraphics[width=0.75\textwidth]{./imagenes/master-dev}
	\caption{Ramas Develop y Master}
\end{figure}
\FloatBarrier

\subsubsection{Ramas Feature}

\quad Cada nueva feature debe residir en su propia rama, que se puede enviar al repositorio central como respaldo/colaboración.\\

\quad En lugar de bifurcarse de master, las features usan el desarrollo como su rama principal, de forma que cuando se completa una característica, se fusiona nuevamente en develop. Las características nunca deberían interactuar directamente con el maestro.\\

\begin{figure}[htb]
	\centering
	\includegraphics[width=1\textwidth]{./imagenes/feature}
	\caption{Rama feature}
\end{figure}
\FloatBarrier

\subsubsection{Ramas Release}

\quad Una vez que el desarrollo ha adquirido suficientes características para un lanzamiento,  bifurca una rama de release fuera del desarrollo. La creación de esta rama inicia el siguiente ciclo de lanzamiento, por lo que no se pueden agregar nuevas características después de este punto. Solo las correcciones de errores, la generación de documentación y otras tareas orientadas a la versión deben ir en esta rama.\\ 

\quad Una vez que está listo para enviar, la release se fusiona en master y se etiqueta con un número de versión. Además, debe fusionarse nuevamente en el desarrollo, que puede haber progresado desde que se inició el lanzamiento.\\

\begin{figure}[htb]
	\centering
	\includegraphics[width=1\textwidth]{./imagenes/release}
	\caption{Rama release}
\end{figure}
\FloatBarrier

\subsubsection{Ramas Hotfix}

\quad Las ramas de mantenimiento o "hotfix" se utilizan para parchear rápidamente las versiones de producción.\\

\quad Las ramificaciones de revisión son muy parecidas a las ramificaciones de lanzamiento y ramificaciones de características, excepto que se basan en master en lugar de develop.\\

\quad Esta es la única rama que debe bifurcarse directamente de master, y tan pronto como se complete la corrección, debe fusionarse tanto en master como en develop (o en la rama de la versión actual), y master debe etiquetarse con un número de versión actualizado.\\

\begin{figure}[htb]
	\centering
	\includegraphics[width=0.6\textwidth]{./imagenes/hotfix}
	\caption{Rama hotfix}
\end{figure}
\FloatBarrier

\newpage

	\thispagestyle{empty} 
	\textcolor[rgb]{1.00,1.00,1.00}{.} 
	\newpage


	%-----------------------------------------------------------------------
	%							BIBLIOGRAFIA
	%-----------------------------------------------------------------------
	
	\begin{thebibliography}{99}
		\bibitem{BBC} 
			\textsc{Los cirujanos que operan a cientos de kilómetros de distancia}
			\textit{BBC}.
			\newline
			\url{https://www.bbc.com/mundo/noticias/2014/05/140520_vert_fut_salud_cirujano_a_distancia_gtg}
		\bibitem{Tech} 
			\textsc{Forgotten genius: the man who made a working VR machine in 1957}
			\textit{techradar}.
			\newline
			\url{https://www.techradar.com/news/wearables/forgotten-genius-the-man-who-made-a-working-vr-machine-in-1957-1318253/2}
		\bibitem{NCSA} 
			\textsc{Virtual Reality: History}
			\textit{NCSA Illinois}.
			\newline
			\url{https://web.archive.org/web/20150821054144/http://archive.ncsa.illinois.edu/Cyberia/VETopLevels/VR.History.html}
		\bibitem{Educativa} 
			\textsc{Gamificación: el aprendizaje divertido}
			\textit{Educativa}.
			\newline
			\url{https://www.educativa.com/blog-articulos/gamificacion-el-aprendizaje-divertido/}
		\bibitem{GameChina} 
			\textsc{‘Gamificación’ de la conducta ciudadana en China}
			\textit{La Razón}.
			\newline
			\url{https://innovadores.larazon.es/es/not/gamificacion-de-la-conducta-ciudadana-en-china}	
		\bibitem{OcuRS} 
			\textsc{Oculus Rift S}
			\textit{Oculus}.
			\newline
			\url{https://www.oculus.com/rift-s/}
		\bibitem{JAlto} 
			\textsc{Qué es la Música 8D y por qué se ha hecho viral}
			\textit{Jaime Altozano}.
			\newline
			\url{https://www.youtube.com/watch?v=e6Ekz7ZDV-w}
		\bibitem{8D} 
			\textsc{Sonido 8D}
			\textit{Verdades y mentiras de la nueva forma de escuchar música que te 'hackea' el cerebro}.
			\newline
			\url{https://www.yasss.es/sabiduria-pop/audio-8d-que-es-hackea-cerebro_0_2643900156.html}
		\bibitem{You} 
			\textsc{Tutorial Retícula}
			\textit{Unity VR Tutorial - Gaze Timer Interaction + Teleport}.
			\newline
			\url{https://www.youtube.com/watch?v=bmMaVTV8UqY}
		\bibitem{Gvr} 
			\textsc{Librería GoogleVR}
			\textit{Unity}.
			\newline
			\url{https://developers.google.com/vr/develop/unity/get-started-android}
		\bibitem{DGvr} 
			\textsc{Descarga GoogleVR}
			\textit{GoogleVR}.
			\newline
			\url{https://github.com/googlevr/gvr-unity-sdk/releases}
		\bibitem{Resonance} 
			\textsc{Librería resonance en unity}
			\textit{Unity}.
			\newline
			\url{https://resonance-audio.github.io/resonance-audio/develop/unity/getting-started}
		\bibitem{DResonance} 
			\textsc{Descarga Resonance}
			\textit{Resonance}.
			\newline
			\url{https://github.com/resonance-audio/resonance-audio-unity-sdk/releases}
		\bibitem{Culling} 
			\textsc{Oclussion Culling}
			\textit{Unity}.
			\newline
			\url{https://docs.unity3d.com/es/current/Manual/OcclusionCulling.html}
		\bibitem{GitKraken} 
			\textsc{GitKraken}
			\textit{Información sobre GitKraken}.
			\newline
			\url{https://support.gitkraken.com}
		\bibitem{Cardboard} 
			\textsc{Cardboard}
			\textit{Información sobre las cardboard}.
			\newline
			\url{https://vr.google.com/intl/es_es/cardboard/}		
		\bibitem{CSharp} 
			\textsc{Guía de C\#}
			\textit{Microsoft}.
			\newline
			\url{https://docs.microsoft.com/es-es/dotnet/csharp/}
		\bibitem{VrEduca} 
			\textsc{VR en las aulas}
			\textit{La realidad virtual en las aulas: ¿Realidad o virtual?}.
			\newline
			\url{https://www.educaciontrespuntocero.com/noticias/realidad-virtual-aulas-educacion/68851.htmll}
		\bibitem{ShadersTut} 
			\textsc{Tutorial shaders}
			\textit{Unity}.
			\newline
			\url{https://docs.unity3d.com/es/current/Manual/Shaders.html}
		\bibitem{GitKraken2} 
			\textsc{GitKraken}
			\textit{Control de Versiones: ¿Por qué GitKraken?}.
			\newline
			\url{https://medium.com/@sergupe6/control-de-versiones-por-qu%C3%A9-gitkraken-ee1f30b4a18f}
		\bibitem{GitFlow} 
			\textsc{Tutorial GitFlow}
			\textit{Atlasian}.
			\newline
			\url{https://www.atlassian.com/git/tutorials/comparing-workflows/gitflow-workflow}		
		\bibitem{RaytracingIntro} 
			\textsc{Raytracing}
			\textit{Introduction to Ray Tracing: a Simple Method for Creating 3D Images}.
			\newline
			\url{https://www.scratchapixel.com/lessons/3d-basic-rendering/introduction-to-ray-tracing?url=3d-basic-rendering/introduction-to-ray-tracing}


	\end{thebibliography}

	
	% \cite{Baz}
	% \vspace{0.06in}

	%\begin{figure}[htb]
	%	\centering
	%	\includegraphics[width=0.4\textwidth]{./imagenes/1}
	%	\caption{Universidad de Granada.} \label{fig:1}
	%\end{figure}


	
\end{document}