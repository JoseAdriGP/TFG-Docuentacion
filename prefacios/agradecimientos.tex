\thispagestyle{empty}

\begin{flushleft}
	\textbf{\LARGE Agradecimientos}\\
\end{flushleft}

\vspace{1cm}

\begin{flushleft}
\quad Empiezo los agradecimeintos con mi tutor Marcelino Cabrera Cuevas, sin el cuál nunca se me habría ocurrido buscar un tfg que planteara algo distinto a la típica interacción con un ordenador de teclado-ratón.\\
	
\quad Quiero hacer hacer una mención especial a mis padres y hemana, ya que sin su constante apoyo no habría podido llevar a cabo mis estudios ni sacar adelante este proyecto.Si estoy donde estoy es también gracias a vosotros.\\
	
\quad Ahora quiero agradecer a mi familia escogida, empezando por Rubén Jiménez Martínez, Vanesa Rodríguez Rodríguez, Nathaniel Jiménez Rodríguez y Juán Manuel Olivero Maroto, cuya amistad ha sido un importante apoyo durante estos años; y continuando por José Pedro Cirre Mateos, David Galindo López, Alén Blanco Domínguez, Michaelle López Eudaric, Benjamín Alba Morales, Miguel Ángel Cano Mesa, Raúl Alberto Calderón López y David Padilla Montero, quienes son antiguos alumnos, grandes amigos y mejores personas.\\
	
\quad Mención también para Juan Hernández García, quien presta su voz para la primera escena de la aplicación y quien es un inestimable amigo, y para José María Esteo Christopoulo, antiguo estudiante de bellas artes y gran amigo con el que he trabajado en unity anteriormente.\\

\quad Agradecimientos también a aquellos profesores que se han dedicado a nosotros sus estudiantes, alentandonos a mantener la curiosidad sobre la profesión que hemos escogido; a mis compañeros de carrera, tanto los que han terminado como los que no, ya que hemos compartidos aulas, alegrías, estrés y sobre todos muchos cafés; y a la universidad, que aunque con el paso de los años sientes que te ha quitado mucho, al terminar te das cuentas que te ha dado mucho más.\\

\quad Por último, quiero hacer una mención especial a Luis Castillo Vidal, que como profesor me animó inconscientemente a seguir adelante con la carrera durante una época peor de mi vida, demostraándome que el que quiere puede.\\
	
\end{flushleft}

\newpage %inserta un salto de página
