\begin{center}
{\large\bfseries VR8D: aplicando sonido 8D a las tecnologías de realidad virtual}\\
\end{center}

\begin{center}
José Adrián Garrido Puertas\\
\end{center}

\begin{flushleft}
	\noindent{\textbf{Palabras clave}: VR, sonido, 8D, Unity, cardboard, eco, posicionamiento, inmersión, oclusion culling, shaders ......}\\
	
	\vspace{0.7cm}
	\noindent{\textbf{Resumen\vspace{0.5cm}}}\\
	En la actualidad, se pueden observar gran variedad de estudios y trabajos relacionados con el concepto de inmersión, buscando con ello facilitar la asimiliación por parte de los usuarios de la información que el programa, terapia, estudio... proporcionan.   
	Con este fin, este proyecto se dispone a investigar la integración en la realidad virtual de los algoritmos de sonido 8D para mejorar la inmersión del usuario en un entorno virtual.
	El trabajo presentado ha sido montado en unity, desarrollándolo como una aplicación en android.
	Las interacciones del usuario con el entorno virtual solo requieren la aplicación, un movil con una cardbord y uno auriculares estereo.
\end{flushleft}

\newpage %inserta un salto de página


