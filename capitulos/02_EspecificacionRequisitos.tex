\section{Especificaciones y Requisitos}

\subsection{Descripción del proyecto}
\subsubsection{Introducción}
\quad Esta sección va a hablar los requisitos principales para el desarrollo de una aplicación que haga uso de los algoritmos de 8D presentados en la actualidad y aplicarlos en un entorno de realidad virtual.\\

\quad El caso presentado comparte similitudes con el desarrollo de un videojuego, las cuales son debidas a que la aplicación en cuestión es un entorno interactivo para poder testear el algoritmo de sonido 8D. Esto se desarrolla en los siguientes apartados.

\subsubsection{Equipo de desarrollo}
\quad Para proyectos de mayor envergadura, el desarrollo de una aplicación de estas características requiere un equipo conformado por los siguientes integrantes:\\
\begin{outline}
\1 Gestión/Producción:
	\2 CEO
	\2 Directores de proyecto
	\2 Productores	
\1 Diseño:
	\2 Game designer
	\2 Level designer
\1 Arte y animación:
	\2 2D artist:
		\3 Concept artist
		\3 Pixel artist
		\3 UI artist
	\2 3D artist (modelaje, iluminación, texturización...):
		\3 Personajes
		\3 Escenario
\1 Animación:
	\2 Rigging
	\2 Animator
\1 Sonido:
	\2 Ingeniero de sonido
	\2 Compositor
\1 Programación:
	\2 Backend
	\2 Frontend
\1 Quality Assurance (QA):
	\2 Testers
	\2 Control de calidad
\end{outline}

\quad En este caso, el equipo está concentrado en una sola persona trabajando todos los aspectos, de forma que se recurre también a assets gratuitos para poder suplir las disciplinas de las que no se tiene conocimiento.\\

\subsubsection{Comparativa de motores gráficos}

\quad Uno de los aspectos más importantes en el desarrollo de esta aplicación es contar con un framework 4 o un motor gráfico sobre el cual poder desarrollarla. A lo largo de
los últimos años se han estandarizado una serie de productos que facilitan este desarrollo.\\

\quad Por ello, se estudiarán diferentes opciones disponibles, teniendo en cuenta que no se pretende señalar todos y cada uno de los motores gráficos del mercado sino una
selección de ellos de acuerdo a mis intenciones de abarcar el máximo espectro posible sin citar a todos.\\

\quad A continuación, se señalan algunos casos:

\begin{itemize}
\item{\textbf{Source 2 Engine}}

\quad Sucesor del motor gráfico Source propiedad de Valve, y motor de varios juegos famosos como Portal o Half Life. Estará disponible de forma pública y gratuita siempre y cuando se publique para la plataforma de Valve, Steam. Compatible con Vulkan (OpenGL) y usa un motor de físicas propio llamado Rubikon que sustituye a Havok. \\

\quad Algunos de los juegos realizados con él son Counter-Strike: Global Offensive y Dota 2.\\

\item{\textbf{Unity}}

\quad Disponible para Windows, OS X y Linux, históricamente asociado a juegos de menor presupuesto, juegos indie y de móviles.\\

\quad Dispone de varias versiones:
\begin{itemize}
	\item Unity Personal: gratuita si no se sobrepasan los 100 mil dólares de ingresos
	\item Unity Plus: suscripción de 35 dólares al año, enfocado a desarrolladores móviles con ingresos menores a 200 mil dólares y que incluyen algunos servicios de Unity
	\item Unity Professional: suscripción de 125 dólares al mes, sin límite de ingresos, con todos los servicios de Unity
	\item Unity Enterprise
	\item Unity Educational
\end{itemize}
\quad Las versiones Pro y Plus ofrecen soporte a la versión y acceso a todas las actualizaciones. También cabe señalar la Asset Store, lugar donde se concentran extensiones, herramientas y assets para los usuarios tanto gratuitos como de pago.\\

\quad Algunos juegos hechos con Unity son Wasteland 2, Pillars of Eternity, Hearthstone o Firewatch.\\

\item{\textbf{CryEngine}}

\quad Desarrollado por Crytek y usado por primera vez en el juego Far Cry. La versión 5 utiliza de forma nativa DirectX12, Vulkan y soporte para VR.\\

\quad Introdujo un nuevo modelo de licencias, el “paga lo que quieras”, 100\% royalty-free en la actualidad, para las plataformas Windows, Linux, PlayStation 4, Xbox One, Oculus Rift, HTC-Vive, Open-Source VR y PlayStation VR.\\

\quad También posee un bazar, el “CRYENGINE Marketplace” donde los beneficios de las ventas son de un 70\% para el motor y el restante 30\% para el desarrollador del contenido.\\

\quad Juegos que usan este motor: Saga Crysis, Ryse: Son of Rome, Aion Online (MMORPG online).\\

\item{\textbf{Unreal Engine}}

\quad Sucesor de Unreal Development Kit (UDK), propiedad de Epic Games. Gratuito en la actualidad, aunque se paga a la empresa un 5\% de los beneficios cada trimestre a partir del momento en que el juego gane sus primeros 3000 dólares. Desarrollado en C++, para su uso en plataformas como Windows, OS X, Linux, iOS, Android, PlayStation 4, Xbox One, Nintendo Switch y navegadores (HTML5). También tiene soporte a Vulkan.\\

\quad Al igual que Unity, tiene su propio bazar llamado “Unreal Engine Marketplace”, donde permite comprar y vender contenido (desde modelos, a tutoriales pasando por sonidos, efectos especiales, etcétera). También ha puesto en marcha un programa llamado “Unreal Dev Grants” con un presupuesto de cinco millones de dólares destinado a financiar a los desarrolladores que presenten proyectos innovadores usando el motor. Probablemente sea el motor más usado en la actualidad.\\

\quad Juegos que usan este motor: DMC (Devil May Cry de Ninja Theory), saga Batman Arkham.\\

\item{\textbf{Snowdrop}}

\quad Motor propiedad de Ubisoft creado por Massive Entertainment (Ubisoft). Codificado en C++, tardó cinco años en ser desarrollado.\\

\quad Su punto fuerte es la iluminación y el sistema de destrucción. Utiliza el motor de físicas Havok. Este motor está en esta lista porque a pesar de ser uno de los mejores del mercado no está disponible para todo el mundo: Sólo los equipos de desarrollo de Ubisoft tienen acceso a este motor.\\

\quad Juegos: Tom Clancy's: The division, Mario + Rabbids Kingdom Battle, Skull \& Bones.\\

\item{\textbf{Frostbite}}

\quad Desarrollado por DICE (Electronic Arts) y diseñado para ser un motor exclusivo de EA. \\

\quad Inicialmente diseñado para hacer juegos en primera persona, ha ido evolucionando abrazando otros géneros. Codificado en C++, C\#, Lua y IronPython. Las plataformas objetivo son PC, PlayStation 4 y Xbox One. Enfocado a permitir una mayor escala de interacciones multijugador en escenarios dinámicamente destruibles con condiciones atmosféricas cambiantes. \\

\quad Como Snowdrop, es un motor propietario. Su presencia aquí es para poder comparar dos motores propietarios. \\

\quad Juegos: Battlefield, Need for speed, Dragon Age, Mirror's Edge, FIFA, Star Wars: Battlefront\\

\item{\textbf{Amazon Lumberyard}}

\quad Motor propiedad de Amazon, gratuito y orientado a juegos AAA (grandes producciones). \\

\quad Basado en CryEngine, está programado en C++ y en Lua. Tiene como características principales la integración con Amazon Web Services (AWS) y Twitch. El código fuente es completo y gratuito y no hay cuotas de suscripción ni requisitos económicos. \\

\quad Solamente hay que pagar por los servicios de AWS que se utilicen (así es como sacan beneficios). Las plataformas objetivo son Windows, PlayStation 4, Xbox One, iOs, Android (con soporte limitado en estas dos últimas), Oculus Rift, HTC-Vive, OpenSource VR y PlayStation VR.\\

\quad Juegos: Star Citizen\\

\item{\textbf{UbiArt framework}}

\quad Motor gráfico propiedad de Ubisoft y diseñado por el creador de Rayman, Michel Ancel. Totalmente centrado en la creación de proyectos 2D y 2.5D, responde a una
búsqueda de la compañía de facilitar el desarrollo de juegos a un equipo pequeño de personas y con un presupuesto reducido.\\

\quad Su punto fuerte es la facilidad para crear animaciones que da a los artistas y el aspecto artístico presente en los títulos desarrollados con este motor.\\

\quad Juegos: Rayman Origins, Rayman Legends, Valiant Hearts, Child of light \\

\item{\textbf{GameMaker}}

\quad Propiedad de YoYo Games y diseñado para permitir a usuarios sin conocimientos de programación desarrollar juegos fácilmente. Contiene un lenguaje de programación de scripts, Game Maker Language (GML), para usuarios experimentados. Licencia EULA.\\

\quad Juegos: Saga Hotline Miami, Undertale.\\
\end{itemize}

\newpage

\subsubsection{Decisión sobre el motor}

\quad A la hora de decir del motor, los factores que han influido son los siguientes:

\begin{itemize}
\item Mínimo coste de obtención de licencia
\item Conocimiento previo.
\item Portabilidad.
\item Escalabilidad.	
\end{itemize}

\quad Hablando de costes, todos tiene la posibilidad de trabajar gratuitamente, pero sólo Unity y Unreal disponen de documentaciones para prácticamente todo lo que se nos pueda ocurrir durante el desarrollo, por lo que hago la discriminación a estos dos.\\ 

\quad Respecto al conocimiento previo, en ambos tengo una larga experiencia, por lo que no es un  factor realmente determinante para excluir definitivamente a ninguno de los dos, aunque con Unity si me he dedicado a trabajar en Game Jams, lo que hace que su interfaz se me antoje más cómoda en una situación de estrés.\\

\quad Lo cierto es que con respecto a la escalabilidad, ambos estarían también empatados, pues ambos disponen de un gran abanico de módulos, plugins y assets externos. Ante esta situación solo puedo determinar que en la escalabilidad, ambos se encuentra igualados.\\

\quad En caso en particular, donde la aplicación será para android, ambos tiene la capacidad de exportarla para este tipo de dispositivos.\\

\quad Finalmente se ha optado por Unity, principalmente por mi anterior experiencia con él en situaciones de estrés trabajando en \textit{Game Jams}.

\subsection{Requisitos}

\quad El servicio que esta aplicación proporciona es simple, ya que se compone de tres entornos virtuales que por el que poder moverte e interaccionar, pero debemos tener en cuenta una serie de requisitos indispensables para el correcto funcionamiento de esta.\\

\quad Dicho esto, nos disponemos a enumerar y describir dichos requisitos para una mejor comprensión por parte del lector de las intenciones de esta aplicación.\\

\subsubsection{Funcionales}

\quad Aquí nos disponemos a hablar de todos aquellos requisitos que la aplicación debe proporcionar directamente. En nuestro caso, estos requisitos son:

\begin{itemize}
	\item Entornos virtuales diferenciados entre sí
	\item Sonido preparado mediante los algoritmos 8D
	\item Interfaz adecuada a la situación planteada
\end{itemize}

\quad Aunque sean pocos, es importante tener en cuenta que cada uno de ellos es imprescindible a la hora del correcto funcionamiento de la aplicación, y por consiguiente, no cumplir uno de ellos invalida la aplicación de forma automática. Así pues, desarrollarlos un adecuadamente.\\

\quad Cuando se habla del \textit{entorno virtual} se quiere hacer referencia al hecho de que necesitamos un entorno por el que el usuario pueda moverse y trabajar con los elementos dentro del entorno. Es importante remarcar que en este proyecto, los entornos van aumentando su complejidad gráfica e interna para así desarrollar una sensación de progresión en el usuario.\\

\quad El sonido es imprescindible para el usuario, ya que proporciona el feedback que se requiere de la aplicación para poder interactuar con los elementos de cada escena. Si el sonido funciona mal, o directamente no funciona, esta aplicación puede considerarse un auténtico fracaso.\\

\quad Al hablar de una interfaz adecuada, se hace referencia al hecho debe cumplir con su propósito y proporcionar feedback al usuario sobre qué se está haciendo con ella. Con respecto a la interfaz hay otros conceptos a tener en cuenta, pero ya entran dentro de los requisitos no funcionales, por lo que los desarrollaremos siguiente apartado,\\

\subsubsection{No funcionales}

\quad Analizar los requisitos que la aplicación debe proporcionar de forma no directa puede ser un poco más problemático, porque dentro de ella se quieren integrar conceptos más abstractos.\\
\quad Para hablar de estos requisitos, vamos a hacer una división sobre el objetivo que pretenden.

\subsubsubsection{Requisitos relacionados con el concepto de inmersión}

\quad Esta sección plantea aquellos requisitos que busca hacer que la experiencia con la aplicación sea lo más inmersiva posible, de forma que el usuario pueda “olvidar” en su subconsciente que se encuentra en un entorno virtual.\\

\quad Conseguir esto requiere varios factores a tener en cuenta, pero el primero sobre el que debemos hablar es la tasa de frames, la cual deberá ser estable dentro de las diferentes escenas. La estabilidad predominará sobre la cantidad de frames ya que se va a trabajar con un smartphone antiguo, y una gran cantidad de frames implicaría también reducir el aspecto gráfico de la aplicación.\\

\quad Otro aspecto a tener en cuenta son los tiempos de carga. El planteamiento de esta aplicación es tener unos tiempos de carga prácticamente inexistentes, de forma que el usuario no sienta una desconexión abisal entre las diferentes escenas.\\

\quad Aquí se vuelve a hablar de los menús, ya que estos no deben estar integrados en el visor del usuario, si no que deben ser elementos que encontremos por el entorno. Este planteamiento se debe a que la intención no es simular que el usuario lleva un casco con una interfaz, si no que se pretende simular que el usuario realmente se encuentra en las escenas presentadas. Un menú que interfiera con la visión rompería la ilusión de verse en otro lugar distinto a la habitación donde se está probando la aplicación.\\

\quad El sonido 8D tiene también un valor añadido en los requisitos no funcionales, ya que este debe proporcionar la posición del emisor de una forma clara, de forma que el usuario sepa en todo momento dónde se sitúa el foco emisor. Esto será útil pues varios de los emisores que se encontrarán en la escena serán también elementos con los que poder interaccionar.\\

\subsubsubsection{Requisitos relacionados con el concepto de gamificación}

\quad Se pretende que parte de las interacciones se produzcan surgidas de un interés relacionado con la gamificación. De está forma, el usuario aprenderá a moverse y a interaccionar con los distintos elementos que componen la escena tomando como referencia los videojuegos.\\



\newpage


