\section{Introducción}

\subsection{Contexto}
\justify 
\quad En la actualidad, vivimos en una sociedad de información, donde cualquier conocimiento (errado o correcto) es fácilmente alcanzable por cualquier usuario desde la palma de la mano. En un mundo como éste, podemos decir que hemos \textit{"sustituído"} nuestra propia memoria por los distintos dispositivos de almacenamiento de los que disponemos diariamente, ya sean móviles, computadores, pen drives...\\
\quad La consecuencia inmediata es que prefiramos demostrar capacidades prácticas en diferentes campos antes que memorizar grandes bloques de información a los cuales tenemos una acceso relativamente sencillo, permitiendonos así consultar los datos concretos que necesitemos en el momento.\\

\subsection{Motivación}
\quad Ultimamente se ha escuchado mucho el concepto de inmersión, tanto aplicado a la enseñanza como a otros campos, lo que me hizo plantearme hasta que punto la inmersión puede afectar al individuo en el ámbito en el que se encuentre. Por ejemplo, hemos visto a lo largo de estos años como cambiando el paradigma de la enseñanza se puede conseguir que el alumno o estudiante absorba mejor los conocimientos, o como la gamificación puede cambiar la conducta del ciudadano (https://innovadores.larazon.es/es/not/gamificacion-de-la-conducta-ciudadana-en-china).\\


\subsection{Objetivo}
\quad Mi objetivo durante este trabajo es hacer la modificaciones pertinentes en el algoritmo de sonido 8D con la intención de mejorarlo, de forma que pueda ser aplicado no solo al estudio, si no al uso práctico en terapias de paliamiento del dolor en enfermedades crónicas, ya que sin poder curar a uina persona con ello, si podemos mejorar su calidad de vida al hacer que puedan durante unos momentos ovidar su condición.
\quad Esta por supuesto, es solo una de las aplicaciones que podemos encontrar para este concepto.\\

\newpage



