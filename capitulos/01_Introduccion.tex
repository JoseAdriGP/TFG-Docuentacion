\section{Introducción}

\subsection{Contexto}
\justify 
\quad En la actualidad, vivimos en una sociedad de información, donde cualquier conocimiento (errado o correcto) es fácilmente alcanzable por cualquier usuario desde la palma de la mano. En un mundo como éste, podemos decir que hemos \textit{"sustituído"} nuestra propia memoria por los distintos dispositivos de almacenamiento de los que disponemos diariamente, ya sean móviles, computadores, pen drives...\\

\quad La consecuencia inmediata es que prefiramos demostrar capacidades prácticas en diferentes campos antes que memorizar grandes bloques de información a los cuales tenemos una acceso relativamente sencillo, permitiendonos así consultar los datos concretos que necesitemos en el momento.\\

\subsection{Motivación}
\quad Últimamente se ha escuchado mucho el concepto de inmersión, tanto aplicado a la enseñanza como a otros campos, lo que me hizo plantearme hasta que puntoesta puede afectar al individuo en el ámbito en el que se encuentre. Por ejemplo, hemos visto a lo largo de estos años como cambiando el paradigma de la enseñanza se puede conseguir que el alumno o estudiante absorba mejor los conocimientos, o como la gamificación puede cambiar la conducta del ciudadano.\\

\quad Como se puede apreciar, ante las personas se presenta una potente \textit{"arma"} de doble filo, la cuál puede hacer mucho bien o mucho mal en función del uso que se le asigne.\\


\subsection{Objetivo}
\quad El objetivo abordado por este proyecto no es otro que hacer la modificaciones pertinentes en el algoritmo de sonido 8D con la intención de mejorarlo, e introducirlo dentro de un estor de realida virtual, de forma que podamos observar el concepto de inmersión en estado puro.\\ 

\quad Si el proyecto finaliza correctamente, el algoritmo demostrará que puede ser aplicado no solo a la aplicación, si no a todo tipo de programas ya sean dedicados a la docencia, estudios variados, uso práctico en terapias de paliamiento del dolor en enfermedades crónicas,...\\

\newpage



