\section{Análisis}

\subsection{Introducción}

\quad Aqui se analizarán las diferentes partes del proyecto agenas a la programación, como la investigación asociada al proyecto, o las herrramientas que se va a utilizar en el proyecto.\\ 

\subsection{Estado del arte}

\quad Aqui se van a analizar varios aspectos sobre el proyecto.\\

	\subsubsection{Sonido 8D}
\quad El audio 8D es la sensación de escuchar los sonidos a través de unos cascos en ángulos de trescientos sesenta grados, parecido a la realidad virtual.\\ 

\quad Este término que a resugido gracias a las redes sociales, dista de ser nuevo, ya que antiguamente era conocido como ambisonic, binaural o simplemente sonido 3D.\\

\quad Como estrategia de márketing, algunos músicos sugieren que la música danza alrededor de quien la escucha, provocando que muchos de los temas actuales, y no tan actuales, adquieran este sistema.\\

\quad A pesar de los avances, la sensación real y envolvente esta muy lejos de la realidad, ya que muchos elementos no se tiene en cuenta a la hora de determinar el posicionamiento del objeto. Por ejemplo, la transición entre delante y detrás del usuario pueden por momentos ser indistinguibles, ya que no se tienen de forma alguna en cuenta la orientación de las orejas, o el hecho de que si el usuario no se mueve, el cerebro humano no es capáz de distinguir entre una posición detante o detrás,asignando por defecto una posición delantera hasta que la fuente del sonido o el receptor se muevan, permitiendose entonces que el cerebro si situe en el espacio virtual.\\

\quad Ante este problema un desarrollador puede verse abrumado al principio, pero la solución a ambos problemas es más simple de lo que podría parecer, ya que en el caso de la orientación de las orejas es aplicar un coeficiente de division en los sonidos situados detrás del usuario, que se basa en el índice de ambsorción de sonido de la piel humana. En este caso se ha utilizado el de la goma para simular la piel. El según do problema no tiene una solución software a simple vista, pero teniendo en cuenta que el usuario interaccionará con el entorno moviendo la cabeza y avanzando por la escena, de forma que cambia su posición y horientación con respecto al foco, este problema se maquillara por la ppropia inmersión de la aplicación.\\ 

	\subsubsection{Realidad Virtual}
\quad Cuando hablamos de realidad virtual, hablamos de un entorno de apariencia real generado mediante tecnología informática que da la sensación de inmersión al usuario.\\ 

\quad Algunos centros educativos lo están usando para comprobar su viabilidad, sobre todo en casos de personas con dificultades en el aprendizaje, aunque presenta una serie de inconvenientes, como el coste o el espacio físico necesario para el usuario. A pesar de ello, se tienen grandes expectativas sobre su utilidad.\\

\quad Es importante añadir que en los últimos años hemos sufrido un boom con la aparición de la VR en el mundo del videojuego de una forma serie, sin embargo el precio de los equipos para los usuarios hace que los desarrolladores no inoven en este campo, por lo que si no se vuelven más económicos, algunos "expertos" afirman que podría desaparecer. \\

\quad Esta afirmación parece muy alarmista, ya que la VR no se utiliza solo en el hámbito del videojuego y se han hecho grandes avances para otros campos.\\

	\subsubsection{Gamificación}
\quad Se como el uso de diseños, elementos y características de juegos en contextos totalmente ajenos a estos. Socialmente hablando, especialmente a aquellos asiduos a las redes sociales, con los que comparten elementos como la lealtad del usuario, los logros o el reclutamiento a eventos con métodos atractivos y divertidos, dándoles una sensación de control y motivando su uso.\\

\quad Respecto a la educación, muchos han adoptado este estilo de enseñanza aunque no todos están de acuerdo. Su uso sigue expandiéndose cada vez más.\\

\quad En 2017, el Gobierno chino puso en marcha un enorme proyecto piloto en el ámbito social, basado en una especie de gamificación. Con ello, se cambió la magnitud y escala de lo que entendíamos por ‘gamificación’ , de tal manera que conectó la actividad online de sus ciudadanos con su estrategia social a gran escala, incorporándolos a un sistema de medición de conducta individual con ‘premios’ y ‘castigos’, a partir de un sistema de puntación llamada originalmente ‘crédito social’ que en origen, tenía propósitos comerciales relacionados con recompensas o incentivos.\\

\quad Esto no sería posible sin dos factores, la colaboración de grandes empresas chinas basadas en internet (como Alibaba, Baidu o Tencent) y las nuevas leyes de ciberseguridad chinas, que dan cobertura legal al acceso completo a casi todos los datos personales.\\

\subsection{Tecnologías y herramientas}

	\subsubsection{Lenguaje de programación C\#}

\quad Es un lenguaje de programación orientado a objetos desarrollado y estandarizado por Microsoft como parte de su plataforma .NET.\\

\quad Al ser un lenguaje orientado a objetos, admite los conceptos de encapsulación, herencia y polimorfismo. Todas las variables y métodos, incluido el método Main, el punto de entrada de la aplicación, se encapsulan dentro de las definiciones de clase. Una clase puede heredar directamente de una clase primaria, pero puede implementar cualquier número de interfaces. Los métodos que invalidan los métodos virtuales en una clase primaria requieren la palabra clave override como una manera de evitar redefiniciones accidentales. En C\#, un struct es como una clase sencilla; es un tipo asignado en la pila que puede implementar interfaces pero que no admite herencia.\\

\quad Escoger este lenguaje facilita el trabajo con unity, motivo por el cuál es el seleccionado.\\
	
	\subsubsection{GitKraken y GitHub}
	
\quad El uso de un repositorio y control de versiones es algo fundamental para llevar un control del trabajo en la aplicación, asi como poder volver a versiones anteriores del desarrollo en caso de ser necesario.\\

\quad GitKraken es una herramienta que tiene absolutamente todas las funcionalidades que se pueden llegar a querer en una herramienta para control de versiones con un rendimiento que deja en muy mal lugar al resto de herramientas (como SourceTree o Tortoise Git).\\

\quad Es la única herramienta que tiene versión de pago con su licencia PRO (49\$/año), sólo necesitarás esta licencia para un uso profesional o no comercial, pudiendo realizar tus proyectos personales con la versión gratuita sin ningún tipo de problemas, ya que las funcionalidades son las mismas.\\

\quad En cualquier caso, con la licencia de estudiante gratuita de GitHub se puede tener acceso a todas las funcionalidades, tanto de GitHub como de GitKraken.\\


	\subsubsection{Unity}

\quad Ha sido la opción para el desarrollo de la app. Ya hablamos de él en el apartado \textit{2.1.3. Comparativa de motores gráficos}.
	
	\subsubsection{Cardboard}

\quad Son un visor donde se coloca un teléfono móvil para experimentar una experiencia en realidad virtual.\\

\quad Actualmente google lo vende como experias en VR a bajo costo.\\

\newpage



